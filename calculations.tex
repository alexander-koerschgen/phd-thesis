\section{Calculations}
We proceed to record calculations on the de Rham-Burnside-Witt complex. We
assume the ground ring to be $\bF_p$, the field with $p$ elements, where $p$ is
at least an odd prime, possibly even greater or equal to 5. We will at first
only consider the case $n=2$. Since we're ultimately interested in $p$-complete
calculations of TC, we restrict ourselves to $p$-adic subgroups given by
diagonalized matrices with entries powers of $p$ (as these are cofinal in the
system of all subgroups).\\
There is not a lot of hope of understanding the Burnside-Witt ring as a ring,
even for the abelian groups $\bZ/p^n\bZ \times \bZ/p^m\bZ$ for $n,m \geq 2$, so
we are going to treat these as underlying black boxes and try to use relations
between the differentials and other structure morphism to gain an understanding
of the dRBW-complex in positive degrees.\\
We are working our way up from the bottom, starting with $G=\{e\}$, the trivial
group, corresponding to $\alpha = \id$, the identity matrix. Here we have $W_G A
= A$. Note that for any commutative ring $R$ with $\bZ \to R$ surjective, the
module of K\"ahler differentials is zero, as the differential is zero on
constants in $R$, i.e. elements of the form $n \cdot 1$ for $n \in \bZ$. This
also holds, by construction, for one-dimensional and thus for all differentials
in the higher de Rham complex on $W_G A = A$, hence $\cW_{\id}\Omega^* \cong 0$
for $* \geq 1$\\
Let $\alpha = \diag(p^n, 1)$ or $\alpha = \diag(1,p^n)$ for some $n \geq 0$.
Then $L_\alpha \cong \bZ/p^n\bZ$ and the Brunside-Witt vectors are isomorphic to
the truncated $p$-typical Witt vectors, yielding $W_\alpha A \cong
\bZ/p^{n+1}\bZ$ \comm{[add reference or make Lemma]}. But then $\bZ \to W_\alpha
A$ is surjective, and as above we have $\cW_{\alpha}\Omega^* \cong 0$ for $*
\geq 1$.\\
In the following discussion, we will readily jump between matrices $\alpha \in
\cM_n$, their kernels $L_\alpha \subset \gT^n$ when interpreted as self-maps of the
torus, and the "standard" p-groups they are isomorphic to. The latter
description will be used to determine the subgroup lattice, the former to apply
relations and structure maps. Note that we will write column vectors as row
vectors as to make the exposition more readable. Note also that we discuss
relations on the generators $\omega_H (a)$ for $H \leq G$ and $a \in A$ (cf.
Rem. \ref{rem_witt_ghost_generators_multiplication}).\\
The next case is $\alpha = \diag(p,p)$ with $L_\alpha = \langle (1/p,0), (0,1/p)
\rangle \cong \bZ/p\bZ \times \bZ/p\bZ \asdef G$. The latter may be interpreted
as $\bF_p^2$, the vector space of dimension 2 over $\bF_p$, and the subgroups
are exactly given as subspaces, which are easily characterized as the two
obvious ones, $0, G$, as well as the one-dimensional ones $H_k \defas \langle
(k,1) \rangle \subset \bZ/p\bZ \times \bZ/p\bZ$ for $k \in \bF_p$ and $H_\infty
\defas \langle (1,0) \rangle$. We have $H_k \cong \langle (k/p,1/p) \rangle =
L_{\beta_k}$ with %
$\beta_k = \left( \begin{smallmatrix} 1 & -k \\ 0 & p \end{smallmatrix} \right)$
	and $\H_\infty \cong \langle (1/p,0) \rangle = L_{\beta_\infty}$ with
	$\beta_\infty = \diag(p,1)$.\\
Our first strategy is to obtain information by pulling elements up from lower
groups, which can be done with the Verschiebung operators 
\begin{displaymath}
  V_\gamma = V_{\beta}^{\gamma\beta}: W_\beta A \to W_{\gamma\beta} A
\end{displaymath}
(multiplying matrices
potentially increases kernels!), using the relation%
\[d_{\gamma v} V_\gamma = V_\gamma d_v\]%
as well as the formula (cf. Def. \ref{def_witt_vers})
\[ V_\gamma: W_H A \to W_G A, \; \omega_K (a) \mapsto \omega_K (a). \]
We utilize that for any cyclic subgroup $H \leq G$, the differentials disappear
on $W_H A$ (which itself is cyclic), hence using the above relation we can
deduce that they disappear for any element pushed up by a Verschiebung
$V_\gamma$ from such a $W_H A$, at least for those differentials that are
indexed by / corespond to vectors in the image of $\gamma$. Indeed, for $x \in
W_H A$ for $H$ cyclic we have
\begin{equation*}
	d_{\gamma v} V_\gamma (x) = V_\gamma d_v (x) = 0,
\end{equation*}
where it suffices to consider $v \in \{e_1,e_2\}$. Hence we just need to
determine $\gamma_H$ for every cyclic subgroup $H$ to obtain the appropriate
relations.\\
Consider $H_\infty \cong L_{\beta_\infty}$. We are looking for $\gamma_\infty$
with $\alpha = \gamma_\infty\beta_\infty$, for then we may use
\begin{equation*}
	V_{\gamma_\infty} = V_{\beta_\infty}^{\gamma_\infty\beta_\infty} = %
	V_{\beta_\infty}^{\alpha}: W_{H_\infty} A \to W_G A.
\end{equation*}
This is given by $\gamma_\infty = \diag(1,p)$, hence we obtain the relations
\begin{equation}\label{eq_group_diag(p,p)_subgroup_diag(1,p)}
	0 = d_1(\omega_{H_\infty}(a)) = p d_2 (\omega_{H_\infty}(a))
\end{equation}
for all $a \in A$, where we used linearity of a differential in its index.\\
Similarly, we have $\gamma_k \defas \inmatrixtwo{p}{k}{0}{1}$ with $\alpha =
\gamma_k \beta_k$, and hence
\begin{equation}\label{eq_group_diag(p,p)_subgroup_(p,k_0,1)}
	0 = p d_1  \omega_{H_k}(a) = (k d_1 + d_2) \omega_{H_k}(a).
\end{equation}
Denoting the trivial subgroup by abuse of notation as $e \defas \{e\} \leq G$,
we observe that $\omega_e(a)$ is in the image of both $V_{\gamma_\infty}$ and
$V_{H_0}$, hence we also obtain
\begin{equation*}
	0 = d_1(\omega_e(a)) = d_2 (\omega_e(a)).
\end{equation*}
We may deduce another relation, using the following: Note that the
multiplication of elements of the form $\omega_G(a)$ is quite simply given by
$\omega_G (a) \omega_G (b) = \omega_G (ab)$, cf. Remark
\ref{rem_witt_ghost_generators_multiplication}, and that $\omega_G(1)$ is the
unit of the multiplication in $W_G A$. Taking an $a \in A = \bF_p$, we have the
identity $a^{p-1} = 1$ (and hence $a^p=a$), and using that $d_i$ is a derivation
for $i \in \{1,2\}$, we get
\begin{gather*}
	d_i(\omega_G(a)) = d_i(\omega_G(a^p)) = d_i(\omega_G(a)^p) = p
	\omega_G(a)^{p-1} d_i(\omega_G(a)) = \\ %
	p \omega_G(a^{p-1}) d_i(\omega_G(a)) = p \omega_G(1) d_i(\omega_G(a)) =
	p d_i(\omega_G(a)). %
\end{gather*}
\comm{[whatever that means!?]} In the not $p$-completed setting, over the usual
torus, everything is just abelian groups, and the differentials are (a priori)
free abelian groups, hence this literally just introduces the relation $(p-1)d_i
(\omega_G(a)) = 0$. In the $p$-completed setting, everything is a module over
the $p$-adic integers. They are a PID. But: modules can have torsion, even over
PID's (duh). So still nothing more to say!? Not quite: have $p d_i (\omega_G(a))
= d_i (\omega_G(a))$, $\bZ_p$ local ring, Nakayama's Lemma implies that the
finitely generated $\bZ_p$ module generated by $d_i (\omega_G(a))$ is zero,
hence $d_i (\omega_G(a)) = 0$. ha! find reference for Nakayama's Lemma (lang?).
Note that the above calculation does not depend on $G$, hence for any $G =
L_\alpha$ we have $d_i(\omega_G(a)) = 0$ for all $i\in \ind{n},\, a \in A$, at
least over $\bZ_p$.\\
Let's move on to the next group.
