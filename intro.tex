% !TEX root = phd_thesis_krasontovitsch.tex
\section{Introduction}
  Why is this interesting? What is some historic background? What are applications? What are you actually doing? Give an outline? What are your results? What more could be done?\\
  Should include: Choice of model of ring spectra and result about equivalence (Shipley - Symmetric spectra and THH) [how much structure do we know to be preserved? comm. ring spectrum (gamma spaces not comm. on the nose, only $E_\infty$), naive equivariant spectrum, structure maps (implied by equivariant structure!?)]; TC, algebraic K-theory and cyclotomic trace (being an equivalence); algebraic version of everything; iteration stuff: red-shift conjecture (rognes), computations (rognes, ausoni);
  \subsection{Acknowledgements}
    Thank people %(supervisors, wife ...?).
  \subsection{Notation}
    We write $\underline{k} \coloneqq \{1 \ldots k \}$. We let $\sS$ and $\sS_*$ denote the category of simplicial sets and pointed simplicial sets, respectively. We refer to the category of (connective) ring spectra, here modeled on $\Gamma$-spaces, as $\bS-\alg$, and to commutative ring spectra by $\bS-\calg$. Given a (pro-finite) group $G$, we write $H \leq G$ whenever $H$ is an (open) subgroup of $G$. Given a morphism between two objects indexed by groups, $X(G) \to X(H)$, we index the morphism $f_G^H: X(G) \to X(H)$, reading the indices bottom to top. When no confusion is possible, we write $T^\alpha \defas [\Lambda_{\sT^n} A]^{L_\alpha}$ for an isogeny of the $n$-torus $\alpha$ and a commutative ring spectrum $A$.
    % TODO add comment about using an object for its identity morphism in notation (in diagrams)?