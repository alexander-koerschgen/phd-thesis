% !TEX root = phd_thesis_krasontovitsch.tex
\section{Iterated THH - Structure Morphisms}
\subsection{Restriction}
Read up in bcd - higher top cyclic hom and  hm - k-theory fin alg over witt...
part of reason for using isogenies: need subgroups $H$ of $G$ such that $G/H$ can be identified naturally with $G$, kernels of surj. homos with finite kernel are an example for that.
\subsection{Frobenius}
\begin{defn}\label{def_frobenius_and_functoriality}
Let $G$ be a group, $A$ a connective commutative ring spectrum, and let $\alpha : G \to G$ be a surjective group homomorphism with finite kernel $L_\alpha$. For every other such morphism $\beta: G \to G$ we define the Frobenius map
\[ F^\alpha \defas F^\alpha_{(\beta)}: \Lambda_G(A)^{ L_{\beta\alpha} } \to \Lambda_G(A)^{ L_\alpha } \]
to be the inclusion of fixed points. This is functorial: Given $\gamma: G \to G$ as above, we have
\[ F^\alpha\beta = F^\beta F^\alpha, \]
or in more detail
\[ F^{\gamma}_{\alpha\beta\gamma} = F^{\gamma}_{\beta\gamma} F^{\beta\gamma}_{\alpha\beta\gamma}. \]

\end{defn}

\subsection{Verschiebung}
\subsection{Teichm\"uller}
\begin{defn}\label{def_Delta_alpha}\cite[Sec. 6.2]{brun2010covering}
\comm{[insert here def of $\Delta_\alpha: A \to T^\alpha$[}
\end{defn}
\begin{prop}\label{prop_iso_degree_0_structure_map_lambda}\cite[Prop. 6.2.4]{brun2010covering}
\comm{[deduce iso $W_G A \to \Lambda_{X} \H A ^G$]}
\end{prop}
\subsection{Differentials}
%
%
\begin{lem}\label{lem_decomp_matrix}\cite[Remark 3.2]{carlsson2011higher}
There is a stable splitting \comm{add $\sigma$ unstable map}%
$$\sT_+^k \simeq \bigvee_{T \subseteq \{1 \ldots k \} } S^{\abs{T}} \cong %
  \bigvee_{j = 0}^k (S^j)^{\vee\binom{k}{j}}.$$
\comm{[geometric torus $(=\bR^n/\bZ^n)$? fix notation!} Given $\alpha \in \M_{n \times k}(\bZ_p),$ this splitting yields the induced stable map of tori
$\alpha_+: (\sT^k_p)_+ \to (\sT^n_p)_+$ as a Matrix $M$ with entries indexed by pairs of subsets
$(S,T)$, $S \subseteq \ind{n}, T \subseteq \ind{k}$ with entries%
$$ M_{S,T} = \left( \sum_{f:T \onto S} \sgn(f) \prod_{j \in T} \alpha_{f(j),j} \right) \eta^{\abs T - \abs S}.$$
We will denote this process with a subindexed asterix, i.e. $(\alpha_+)_*$.% bazinga change this!!
\end{lem}
%
%
\begin{lem}\label{lem_decomp_matrix_funct}
Assuming that $\eta$ is nullhomotopic, the above splitting is functorial with respect to matrix multiplication.
\begin{proof}
Let $a \in \M_{n \times k}(\bZ_p), b \in \M_{m \times n}(\bZ_p)$, and set $A \defas (a_+)_*$, $B \defas (b_+)_*$. Observe first that $A_{S,T}$ is zero unless $S$ and $T$ have the same cardinality: If $\abs{S} > \abs{T}$, the sum is empty; if $\abs{S} < \abs{T}$, we have a positive power of $\eta$, which by assumption is zero. Observe further that for $\abs{S} = \abs{T}$, we have $A_{S,T} = a_{S,T}$, the $(S,T)$-minor of $a$.\\
Now we compute:
$$(BA)_{S,T} = \sum_{X \subseteq \ind{n}} B_{S,X} A_{X,T} = \sum_{X \subseteq \ind{n}, \abs{X} = k} b_{S,X} a_{X,T},$$
where $k = \abs{S} = \abs{T}$ and $(BA)_{S,T} = 0$ if $\abs{S} \neq \abs{T}$. On the other hand,
$$((ba_+)_*)_{S,T} = (ba)_{S,T}$$ and these two terms are equal by the Binet-Cauchy formula from linear algebra, proving that%
$$ (b_+)_*(a_+)_* = (ba_+)_*.$$
\end{proof}
\end{lem}
%
%
\begin{lem}\label{lem_decomp_mult_2}\cite[Lemma 3.1]{carlsson2011higher}
The multiplication on the $1$-torus $\sT^1_+ \wedge \sT^1_+ \to \sT^1_+$ with respect to the stable splitting is given by the matrix
\begin{equation}\label{eq_torus_mult_2}\mu_* = %
\left( \begin{array}{cccc}
1 & 0 & 0 & 0 \\
0 & 1 & 1 & \eta \\
\end{array} \right): S^0 \wedge S^1 \wedge S^1 \wedge S^2 \to S^0 \wedge S^1 .\end{equation}
\end{lem}
%
%
%\begin{cor}\label{cor_decomp_mult_n}
%Assuming that $\eta$ is nullhomotopic, the multiplication on the $n$-torus $\mu:\sT^n_+ \wedge \sT^n_+ \to \sT^n_+$ with respect to the basis obtained from the stable splitting has the following description. Let $A,B,C \subseteq \ind{n}$. Then
%$$(\mu_*)_{(A,B),C} = \twopartdef %
%{\sgn(A \cup B)\cdot\eta^k}		{A \cup B = C}%
%{0}			{\mathrm{else}}$$%
%where $k = \# A \cap B \cap C$.
%\begin{proof}
%We identify $$\sT^n_+ \wedge \sT^n_+ \simeq \bigvee_{X \subseteq \{1 \ldots n \} } S^{\abs{X}} \wedge \bigvee_{Y \subseteq \{1 \ldots n \} } S^{\abs{Y}} \simeq \bigvee_{T \subseteq \{1 \ldots 2n \} } S^{\abs{T}}$$ via the map $\psi: \cP(n) \x \cP(n) \to \cP(2n)$ sending $(A,B) \longmapsto A \cup (n + B)$, where $(n+i) \in (n+B)$ iff $i \in B$, and index %
%$$ \bigwedge_{i=1}^n (S^0 \vee S^1) \simeq \bigvee_{T \subseteq \ind n} S^{\abs T} \simeq \sT^n_+$$ by choosing $S^1$ in the $i$-th smash-factor iff $i \in T$. Keeping these indexings in mind, one obtains the above formula for the $n$-fold tensor product of the matrix from equation (\ref{eq_torus_mult_2}).
%\end{proof}
%\end{cor}
%
%
\begin{defn}\label{def_diffs} %get the p out? bazinga
Let $A$ be a (naive) left $\sT^n$-spectrum. Note that $T^n = \Sigma^\infty T^n \in \Sp$ is semistable, hence given a stable class $x \in \pi^s_k(T^n)_+$, we may choose a representative $f: S^k \wedge S^m \to \sT^n \wedge S^m $\\
%
\comm{[can $m$ be chosen to be $k$ (relevant?)? does this definition agree with hesselholt?]}\\
%
of $x = [f]$, and define the operator
	\[	d_x: \pi_0 A \to \pi_k A	\]
as the composite
	\[	\pi_0 A \to%
		%\xrightarrow{\mathmakebox[1.8\trollo] {S^m \wedge S^k \wedge \blank} } %
		\pi_{k+m} ( S^k \wedge S^m \wedge A ) %
			\xrightarrow{\mathmakebox[1.8\trollo]{( (\tau \circ f) \wedge A)_*}} %
		\pi_{k+m} ( S^m \wedge T^n \wedge A )
			%
	\]
	\[
			\xrightarrow{\mathmakebox[1.4\trollo]%
				{(S^m \wedge \mu)_*}} %
		\pi_{k+m} ( S^m \wedge A ) %
			\myrightarrow{\chi_{k,m}} %
		\pi_{m+k} (S^m \wedge A ) %
			\to
		\pi_k A,	\]
where the first and last morphism are the natural suspension isomorphism of stable homotopy groups $ S^l \wedge \blank: \pi_k(X) \to \pi_{l+k} (S^l \wedge X)$ and its inverse, respectively, $\mu: \sT^n \wedge A \to A$ is the action map, $\tau: X \wedge Y \to Y \wedge X$ is the twisting map and $\chi_{k,m} \in \Sigma_{k+m}$ is the shuffle permutation permuting the block of the first $k$ numbers past the block of the last $m$ numbers, acting on $\pi_{k+m}$ by permuting the coordinates in the source sphere. Since we examine the effect of $f$ on stable homotopy groups, and since (regarding the suspension isomorphisms) the smash product is associative, this is independent of choice of representative. We will often blur the distinction between a representative $f$ and its class $[f]=x$.\\
We restrict ourselves to a certain class of maps: Recall the stable splitting of the torus (cf. Lemma \ref{lem_decomp_matrix}). Assuming that $\eta$ is trivial, we only allow morphisms%
\[f: S^k \to (\sT_p^n)_+ \simeq \bigvee_{T \subseteq \{1 \ldots n \} } S^{\abs{T}}\] with $(f_*)_T = 0$ for all $\abs{T} \neq k$, or equivalently $\pi_*(f)=0$ for all $* \neq k$, and denote this $\bZ_p$-submodule of $\SHC(S^k,(\sT^n_p)_+)$ as $C_k$. Their collection forms the graded $\bZ_p$-submodule $C_* \subseteq \SHC(S^*,(\sT^n_p)_+)$, where the $\bZ_p$-module structure comes from the isomorphism $\pi_0(\sT_p^n) \cong \bZ_p$. % bazinga [actually, for prime 2, need to include powers of $\eta$, they come up when multiplying elts (CHECK)]
Note that this extends \cite[Definition 3.3]{carlsson2011higher}, which considers maps $$S^k \myrightarrow{\sigma} (\sT_p^k)_+ \myrightarrow{\alpha_+} (\sT_p^n)_+$$ where $\alpha:\bZ_p^k \to \bZ_p^n$ is a matrix and $\sigma$ refers to the stable splitting of the torus. In particular for $n=1$ the two definitions coincide.
Furthermore we have $C_k = 0$ for $k \notin \{0 \ldots n\}$. % bazinga [allowing powers of $\eta$ admits morphisms from further up, are they automatically zero?]
\end{defn}
%
%
\begin{prop}\label{prop_diff_derivations}
One-dimensional differentials are derivations, i.e. satisfy the Leibniz rule...
\end{prop}
\begin{proof}
\comm{write precise statement, write proof}
\end{proof}
%
\begin{lem}\label{lem_diff_alt_alg}
Assume that $p$ is an odd prime. The collection $C \defas (C_k)_{k \in \bZ}$ of maps indexing the differentials form a free exterior algebra over $\bZ_p$ with generators $e_i: S^1 \to (\sT_p^n)_+$ for $i \in \ind{n}$, where $e_i \defas e_{\{i\}}$ and (more generally) $e_A: S^{\abs{A}} \to (\sT_p^n)_+$ becomes the identity after projecting onto the $A$-th summand for $A \subseteq \ind{n}$. The product is given by the smash product of two morphisms followed by postcomposition with the multiplication on $\sT_p^n$:
$$\SHC(S^k, (\sT_p^n)_+) \otimes \SHC(S^l, (\sT_p^n)_+) \myrightarrow{\wedge}%
\SHC(S^{k+l}, (\sT_p^n)_+ \wedge (\sT_p^n)_+) \myrightarrow{\mu_*}%
\SHC(S^{k+l}, (\sT_p^n)_+).$$
\begin{proof}
We first show that $C_ *$ is isomorphic to the free graded $Z_{(p)}$-module generated by $e_A$ (in degree $\abs{A}$) for $A \subseteq \ind{n}$: We identify%
$$\SHC(S^k, (\sT_p^n)_+) \myrightarrow{\simeq} \pi_k((\sT_p^n)_+)$$%
via evaluation at the fundamental class $\iota_k \in \pi_k(S^k)$ and further identify via the stable splitting%
$$(\sT_p^n)_+ \simeq \bigvee_{T \subseteq \ind{n}} S_p^{\abs{T}}.$$%
By definition of $C_k$, postcomposition with the map induced by the projection onto $k$-dimensional summands%
$$\bigvee_{T \subseteq \ind{n}} S_p^{\abs{T}} \to \bigvee_{T \subseteq \ind{n}, \abs{T}=k} S_p^{\abs{T}}$$%
yields an isomorphism%
$$C_k \to \pi_k  \bigvee_{T \subseteq \ind{n},\abs{T}=k} S_p^{\abs{T}} .$$
Finally we compute
$$\pi_k  \bigvee_{T \subseteq \ind{n},\abs{T}=k} S_p^{\abs{T}} \cong %
    \bigoplus_{T \subseteq \ind{n},\abs{T}=k} \pi_k S_p^k \cong
    \bigoplus_{T \subseteq \ind{n},\abs{T}=k} \bZ_p,$$%
using that localization commutes with homotopy groups for symmetric spectra whose homotopy groups are finitely generated, which is the case for $S^k$.

We proceed to prove that for $A,B \subseteq \ind{n}$ we have $e_A \cdot e_B = 0$ if $A \cap B \neq \varnothing$ and $e_A \cdot e_B = \sgn(A \cup B) e_{A\cup B}$ otherwise, where the signum is taken from the permutation bringing the tupel $(AB)$ into ascending order. We use the splitting%
$$\sT^n_+ \simeq \bigwedge_{ A \subseteq \ind{n} } S^{\abs{A}} \simeq %
\bigwedge_{ i \in \ind{n} } (S^0 \vee S^1),$$
where we recall the last identification: The brackets on the right are resolved, and we obtain $S^{\abs{A}}$ by choosing $S^1$ in the $i$-th bracket if $i \in A$, and $S^0$ otherwise. We also recall Equation \ref{eq_torus_mult_2}, which describes the multiplication in terms of said splitting for $n=1$:
\begin{equation*}\mu_* = %
\left( \begin{array}{cccc}
1 & 0 & 0 & 0 \\
0 & 1 & 1 & \eta \\
\end{array} \right): S^0 \wedge S^1 \wedge S^1 \wedge S^2 \to S^0 \wedge S^1 .\end{equation*}
Let $A, B \subseteq \ind{n}$, $A = \{i_1, \ldots, i_k\}$, $B = \{j_i, \ldots, j_l \}$ with $i_1 \less \ldots \less i_k$, $j_1 \less \ldots \less j_l$. We analyze the following diagram:
\[
\xymatrix{
  S^{\abs{A}} \wedge S^{\abs{B}} \ar[r]^-{e_A \wedge e_B} \ar[dr] & %
  (\sT_p^n)_+ \wedge (\sT_p^n)_+ \ar[r]^(.55){\mu} \ar[d]^{\simeq} & %
  (\sT_p^n)_+ \ar[d]^{\simeq} \\ %
    %
  & %
  \bigwedge\limits_{i \in \n} (S^0 \vee S^1) \wedge %
    \bigwedge\limits_{i \in \n} (S^0 \vee S^1) \ar[r] & %
  \bigwedge\limits_{i \in \n} (S^0 \vee S^1).
}
\]
Since the multiplication is defined entrywise, it becomes a smash-product of one-dimensional multiplications in the lower row, represented by the matrix above. Considering the $j$-th factor in the products in the lower row, the matrix decodes to the following: We get an identity if the target is $S^0$ and both sources are $S^0$, or if the target is $S^1$ and exactly one of the two sources is $S^1$. All other cases lead to zero (recall that $\eta$ is zero, since $p$ was chosen to be odd). So we need only look at the target sphere corresponding to $A \cup B$, and we may assume that $A \cap B = \varnothing$.

Let $x \wedge y = (x_{i_1} \ldots x_{i_k}) \wedge (y_{j_1} \ldots y_{j_l}) \in S^{\abs{A}} \wedge S^{\abs{B}}$. Then the diagonal arrow takes this element to $ \tilde{x} \wedge \tilde{y}$, where $\tilde{x}_i = x_i$ if $i \in A$, and $\tilde(x)_i = \ast$, the non-basepoint of $S^0$, otherwise; $\tilde{y}$ is defined analogously. This, in turn, is taken to the element $z = (z_1 \ldots z_n)$ with
\[ z_i = \left\{
		\begin{array}{ll}
			\ast & i \nin A \cup B \\
			x_i & i \in A \\
			y_i & i \in B
		\end{array}
	\right.\]
hence the composotion $S^{\abs{A}} \wedge S^{\abs{B}} \to S^{A \cup B}$ has degree signum of the permutation bringing $(AB)$ into ascending order, and we obtain the claimed formula $e_A \cdot e_B = \sgn{(AB)} \cdot e_{A \cup B}$ for disjoint $A,B$.
\end{proof}
\end{lem}
%
%
\begin{cor}\label{cor_diffs_ext_alg}
Let $X$ be a $\sT_p^n$-spectrum. The morphism of graded abelian groups%
  \[C_* \to \SHC(S^* \wedge X, X)\]%
sending a map in $C_k$ to the corresponding differential $X \wedge S^k \to X$ is a morphism of graded rings, where multiplication of differentials is given by composition.
\begin{proof}
The assignment sending $f \in C_k$ to $\mu \circ (f \wedge X):S^k \wedge X \to X$, where $\mu$ refers to the action map $(\sT_p^n)_+ \wedge X \to X$, is by definition bijective and additive as a composition of additive maps. The multiplicativity follows from associativity of the torus action as well as functoriality of the smash product in both variables.
\end{proof}
\end{cor}
%
%
