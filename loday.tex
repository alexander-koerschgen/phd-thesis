% !TEX root = phd_thesis_krasontovitsch.tex
\section{Iterated THH - Loday Functor}
  We proceed to introduce topological Hochschild homology based on a space $X$ and the structure it carries for $X = T^n$ the $n$-dimensional torus, following \cite{brun2010covering} as well as \cite{carlsson2011higher}. For details on bicategories, confer \cite{benabou1967introduction}.\\
  \begin{defn}\label{def_cateogry_of_spans}
    The category of spans $V$ ...
  \end{defn}
  \begin{defn}\label{def_bicategory}
    A bicategory $\mathcal{C}$ consists of the following data:
  \end{defn}
  \begin{ex}\label{ex_bicategory_of_categories}
    The strict bicategory of small categories
  \end{ex}
  \begin{defn}\label{def_bicategory_functor}
    A functor $F: \mathcal{C} \to \mathcal{D}$ is a weak / strict / lax / ? functor...
  \end{defn}
  \begin{defn}\label{def_bicategory_of_spans}
    The bicategory of spans $W$ ...
  \end{defn}
  \begin{defn}\label{def_subcategories_of_epimorphisms_and_isomorphisms}
    subcategories of epimorphisms and isomorphisms of spans;
  \end{defn}
  \begin{defn}\label{def_lax_functor_V_to_eW}
    the lax functor from bicategory of spans to subcategory of epimorphisms
  \end{defn}
  \begin{defn}\label{def_functors_finite_sets_to_iW}
    functors from finite sets and finite sets op to $iW$: composition of inclusion into $V$ and $L$, actually lands in $iW$.
  \end{defn}
  \begin{defn}\label{def_functor_iW_to_Cat}
    ...
  \end{defn}
  \begin{defn}\label{def_pointed_stuff}
    pointed category, functor, smash, adjoining basepoint to category,
  \end{defn}

  A bicategory $\mathcal{C}$ is in part made up of a class of 0-cells, and for any two zero-cells $A,B$ a category $\mathcal{C}(A,B)$, whose objects form the 1-cells from $A$ to $B$ and whose morphisms form the 2-cells between two given 1-cells. The bicategory of spans $W$ has 0-cells all finite cells. Given finite sets $X,Y$ the 1-cells are given as spans $ X \leftarrow A \rightarrow Y$ for some finite set $A$, and a 2-cell between two spans from $X$ to $Y$  is given as the vertical map in the following commutative diagram:
  \[
  \xymatrix@R-=.5em{
    %
    &
    %
    A \ar[dl] \ar[dd] \ar[dr]
    %
    \\
    X
    &
    %
    &
    Y
    \\
    %
    &
    A^\prime \ar[ul] \ar[ur]
    &
    %
  }
  \]
  Horizontal composition is given by a functorial and conrete choice of pullback applied to the 1-cells and taking the map induced by the 2-cells between pullbacks, \comm{[make clearer or scratch - this should explain horizontal composition of 2-cells!]}, while vertical composition is composition of maps.\\
  The bicategory $\operatorname{Cat}$ of small categories has small categories as 0-cells, functors as 1-cells and natural transformations as 2-cells.
  \comm{add all the technical things you need from covering homology:}\\
  spans, functor $\cJ$, nat traf $G^A_S$ (gamma spaces, hom space (fibrant replacement), $(\Lambda_X A)^G$ functor of conn. comm $S$-algebras that preserves conn. and has values in very special gamma spaces (Cor. 5.1.5 in Covering homology), how diagonal is constructed (Street rectification necessary! H-set, and so on...); adapt to orthogonal spectra!?
  \begin{defn}\label{def_loday_functor}
    We define the Loday functor for a finite set $S$ and a commutative $\bS$-algebra $A$ as hocolim category functor ...

  \end{defn}

  \begin{defn}\label{def_loday_functor_in_symmetric_monoidal_category}
    Let $(\mathcal{C},\smash, \mathbb{1})$ be a symmetric monoidal category.\\
    \comm{TODO complete this}
  \end{defn}

  \begin{lem}\label{lem_loday_functor_is_simplicial}
    The Loday functor is a simplicial functor.\\
    \comm{[loday functor simplicial! add remark? add proof? source ($\Gamma$-spaces)?]}
  \end{lem}
