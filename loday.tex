% !TEX root = phd_thesis_krasontovitsch.tex

\section{Iterated THH - Loday Functor}
  We proceed to introduce topological Hochschild homology based on a space $X$ and the structure it carries for $X = T^n$ the $n$-dimensional torus, following \cite{brun2010covering} as well as \cite{carlsson2011higher}. For details on bicategories, confer \cite{benabou1967introduction}.\\
  \comm{[Should have two subsections here: spans + bicategories?]}

  \subsection{Spans of finite sets}

  \begin{defn}\label{def_cateogry_of_spans}
    The category of spans $V$ has as objects the class of finite sets, while the morphism set $V(Y,X)$ for two finite sets $X,Y$ is given by the set of equivalence classes of diagrams of the form $Y \ot A \to X$, called spans. Two such diagrams $Y \ot A \to X$, $Y \ot A^\prime \to X$ are said to be equivalent if there is a bijection $A \to A^\prime$ making the resulting triangles commutative. To compose two maps represented by $[Y \ot A \to X]$ and $[X \ot B \to W]$, we take the pullback $A \ot C \to B$ of $A \to X \ot B$ and compose with the maps $A \to Y$, $B \to W$ to obtain $[Y \ot C \to W]$.
    % Since we take equivalence classes of morphisms, this is well-defined, and it is straight-forward to check that this forms a category.
  \end{defn}

  \begin{lem}\label{lem_coproduct_product_in_V}
    The product and coproduct in V are given by disjoint union.
    % \begin{proof}
      % Explain structure morphisms, comment on / explain universal properties
    % \end{proof}
  \end{lem}

  \begin{defn}\label{defn_group_acting_on_object_in_V_morphism_phi}
    Given a finite set $X$ with a left action of a group $G$, we define the action on $X$ through automorphisms in $V$ by mapping $g \mapsto [X \myleftarrow{\id_X} X \myrightarrow{g} X]$, which can easily be seen to be a left action. Thus $G$ also acts (from the left) on $V(Y,X)$ for any finite set $Y$, by functoriality. We define a function
    \begin{gather*}
      \phi: V(Y,X/G) \to V(Y,X)^G \\
      [Y \ot A \to X/G] \mapsto [Y \ot A \x_{X/G} X \to X].
    \end{gather*}
    This function is bijective.
  \end{defn}

  % \begin{prop}\label{prop_phi_V(Y,X/G)_to_V(Y,X)^G_is_bijective}
  %   The function $\phi$ defined above is bijective.
  %   \begin{proof}
  %
  %   \end{proof}
  % \end{prop}

  \begin{defn}\label{def_bicategory}
    A bicategory $\cC$ consists of the following data:
    \begin{itemize}
      \item A class of objects, or 0-cells, of $\cC$, denoted $\cC_0$;
      \item For any two objects $c, d \in \cC_0$ a category $\cC(c,d)$ whose objects $c \myrightarrow{f} d$ are called 1-morphisms, or 1-cells of $\cC$, and whose morphisms
      \begin{displaymath}
        \begin{xy}
          \xymatrix{
            c 
              \ar@/_1pc/[r]_{g}^{}="g" 
              \ar@/^1pc/[r]^{f}_{}="f" 
            &
            d
            \ar@{=>} "f";"g" ^{\alpha}    
          }
        \end{xy}
      \end{displaymath}
      are called 2-morphisms, or 2-cells of $\cC$.
      \item For any three objects $c,d,e \in \cC_0$ a functor $\cC(d,e) \times \cC(c,d) \to \cC(c,e)$, called composition, and written as
      \begin{gather*}
        (g,f) \mapsto g \circ f \asdef gf,\\
        (\beta,\alpha) \mapsto \beta \ast \alpha,
      \end{gather*}
      where $f,g$ are 1-morphisms and $\alpha,\beta$ are 2-morphisms, or in diagram form:
      \begin{gather*}
        \xymatrix{
          c% 
            \ar@/_1pc/[r]_{f^\prime}^{}="fprime" 
            \ar@/^1pc/[r]^{f}_{}="f" 
              \ar@{=>} "f";"fprime" ^{\alpha}
          &
          d% 
            \ar@/^1pc/[r]^{g}_{}="g" 
            \ar@/_1pc/[r]_{g^\prime}^{}="gprime"
              \ar@{=>} "g";"gprime" ^{\beta}
          &
          e%
            \ar@{|->}[r]
          &
          c%
            \ar@/^1pc/[rr]^{g \circ f}_{}="gf" 
            \ar@/_1pc/[rr]_{g^\prime \circ f^\prime}^{}="gfprime"
              \ar@{=>} "gf";"gfprime" ^{\beta \ast \alpha}
          &&
          e% 
        }
      \end{gather*}
      \item For each object $c \in \cC_0$ an 1-morphism $1_c \in \cC(c,c)$, taking the role of the identity on $c$;
      \item For each triple $a\myrightarrow{f}b\myrightarrow{g}c\myrightarrow{h}d$ of 1-morphisms an isomorphism in the category $\cC(a,d)$%
      \begin{displaymath}
        \begin{xy}
          \xymatrix{
            a \ar@/_1.5pc/[rr]_{h \circ (g \circ f)}^{}="g" 
              \ar@/^1.5pc/[rr]^{(h \circ g) \circ f}_{}="f" 
            && 
            d
              \ar@{=>} "f";"g" ^{\alpha_{h,g,f}}
          }
        \end{xy}
      \end{displaymath}
      called the associativity coherence isomorphism;
      \item For each 1-morphism $c \myrightarrow{f} d$, two isomorphisms in $\cC(c,d)$
      \begin{displaymath}
         \begin{xy}
          \xymatrix{
            c \ar@/_1.5pc/[rr]_{f}^{}="f1"
              \ar@/^1.5pc/[rr]^{1_d \circ f}_{}="1_d \circ f" 
            && 
            d
              \ar@{=>} "1_d \circ f";"f1" ^{\lambda_f}
            &&
            c \ar@/_1.5pc/[rr]_{f}^{}="f2"
              \ar@/^1.5pc/[rr]^{f \circ 1_c}_{}="f \circ 1_c" 
            && 
            d
              \ar@{=>} "f \circ 1_c";"f2" ^{\rho_f}
          }
        \end{xy}
      \end{displaymath}
      the unit coherence isomorphisms, indicating left and right cancellation of the unit 1-arrows.
    \end{itemize}
    This data is subject to the following relations: 
    \begin{itemize}
      \item The coherence isomorphisms $\alpha_{f,g,h}$, $\lambda_f$ and $\rho_f$ are natural in $f,\,g,\,h$, and $f$, respectively;
      \item Given four composable 1-arrows $\bullet\myrightarrow{f}\bullet\myrightarrow{g}\bullet\myrightarrow{h}\bullet\myrightarrow{i}\bullet$, we have the following commutative diagram (called pentagon axiom)
      % TODO port the \scriptsize to other places in relations where you used \txt
      \begin{displaymath}
        \begin{xy}
          \xymatrix@C=0.2em{
            %
            &
            %
            &
            (i \circ h) \circ (g \circ f)
              \ar@{=>}[drr]^-*!/ur 3pt/\txt<50pt\scriptsize>{$\alpha_{i,h,g \circ f}$}
            &
            %
            &
            %
            \\
            ((i \circ h) \circ g) \circ f
              \ar@{=>}[urr]^-*!/ul 3pt/\txt<50pt\scriptsize>{$\alpha_{i \circ h,g,f}$}
              \ar@{=>}[dr]_-*!/dl 3pt/\txt<50pt\scriptsize>{$\alpha_{i,h,g} \ast \id_f$}
            &
            %
            &
            %
            &
            %
            &
            i \circ (h \circ (g \circ f))
            \\
            %
            &
            (i \circ (h \circ g)) \circ f
              % \ar@{=>}[rr]_{\alpha_{i,h \circ g, f}}
              \ar@{=>}[rr]_-*!/d 3pt/\txt<50pt\scriptsize>{$\alpha_{i,h \circ g, f}$}%
            &
            %
            &
            i \circ ((h \circ g) \circ f)
              % \ar@{=>}[ur]_{\id_i \ast \alpha_{h,g,f}}
              \ar@{=>}[ur]_*!/dr 3pt/\txt<50pt\scriptsize>{$\id_i \ast \alpha_{h,g,f}$}
            &
            %
            }
        \end{xy}
      \end{displaymath}
      where the 2-arrow $\id_f$ is the identity morphism on the object $f$ in its morphism category.
    \item Given two composable 1-arrows $\bullet\myrightarrow{f}\bullet\myrightarrow{g}\bullet$, we have the following commutative diagram (called triangle axiom)
          \begin{displaymath}
        \begin{xy}
          \xymatrix@C=0.2em{
            (g \circ 1_d) \circ f
              \ar@{=>}[rr]^-*!/u 2pt/\txt<50pt\scriptsize>{$\alpha_{g,1_d,f}$}%
              % \ar@{=>}[rr]^{\alpha_{g,1_d,f}}%
              \ar@{=>}[dr]_{\rho_g \ast \id_f}
            &
            %
            &
            g \circ (1_d \circ f)
              % \ar@{=>}[dl]^*!/dr 3pt/\txt<50pt\scriptsize>{$\id_g \ast \lambda_f$}
              \ar@{=>}[dl]^{\id_g \ast \lambda_f}
            \\
            %
            &
            g \circ f
            &
            %
            }
        \end{xy}
      \end{displaymath}
    \end{itemize}
    A bicateory is called \emph{strict} if all coherence isomorphisms are identities.
  \end{defn}

  \begin{rem}
    The composition of 2-arrows induced by the composition functor is denoted by an asteriks as 2-arrows enjoy an internal composition stemming from the fact that they are morphisms in a category, namely $\cC(c,d)$ for two objects $c,\,d$ of a bicategory $\cC$. Drawing out the diagrams involved in the definition of functoriality for the composition functor,
    \begin{gather*}
        \xymatrix{
          c% 
            \ar@/^2pc/[rr]^{f}_{}="f" 
            \ar[rr]^{}="f_prime_upper"^(.65){f^\prime}_{}="f_prime_lower"
            \ar@/_2pc/[rr]_{f^\dprime}^{}="f_dprime" 
              \ar@{=>} "f";"f_prime_upper" _{\alpha}
              \ar@{=>} "f_prime_lower";"f_dprime" ^{\alpha^\prime}
          &
          %
          &
          d% 
            \ar@/^2pc/[rr]^{g}_{}="g"
            \ar[rr]^{}="g_prime_upper"^(.65){g^\prime}_{}="g_prime_lower"
            \ar@/_2pc/[rr]_{g^\dprime}^{}="g_dprime"
              \ar@{=>} "g";"g_prime_upper" _{\beta}
              \ar@{=>} "g_prime_lower";"g_dprime" ^{\beta^\prime}
          &
          %
          &
          e%
            \ar@{|->}[r]
          &
          c%
            \ar@/^2pc/[rr]^{g \circ f}_{}="gf"
            \ar[rr]^{}="gf_prime_upper"^(.7){g^\prime \circ f^\prime}_{}="gf_prime_lower"
            \ar@/_2pc/[rr]_{g^\dprime \circ f^\dprime}^{}="gf_dprime"
              \ar@{=>} "gf";"gf_prime_upper" _{\beta \ast \alpha}
              \ar@{=>} "gf_prime_lower";"gf_dprime" ^(.4){\beta^\prime \ast \alpha^\prime}
          &&
          e,% 
        }
      \end{gather*}
    we see that functoriality of the composition functor implies commutativity of the following diagram
    \begin{gather*}
      \xymatrix{
        g \circ f
          \ar[r]^{\beta \ast \alpha}
          \ar[dr]_(.4){(\beta^\prime \circ \beta) \ast (\alpha^\prime \circ \alpha)}
        &
        g^\prime \circ f^\prime
          \ar[d]^{\beta^\prime \ast \alpha^\prime}
        \\
        %
        &
        g^\dprime \circ f^\dprime,
      }
    \end{gather*}
    or in formulas
    \begin{displaymath}
      (\beta^\prime \ast \alpha^\prime) \circ (\beta \ast \alpha) = (\beta^\prime \circ \beta) \ast (\alpha^\prime \circ \alpha)
    \end{displaymath}

    demanding compatibility of the internal composition with the external composition of 2-morphisms.
  \end{rem}

  \begin{ex}\label{ex_bicategory_of_categories}
    The strict bicategory $\Cat$ of small categories has as objects all small categories, while the morphism category $\Cat(\cC,\cD)$ is the category of functors and natural transformations between the two categories.
    % TODO add explicit description of composition functor? two choices, both equal because of naturality; proof of functoriality? have notes, see pic on phone (A3)
  \end{ex}

  \begin{defn}\label{def_bicategory_functor}
    A functor $F: \mathcal{C} $% \mathcal{D}$ is a weak / strict / lax / ? functor...
  \end{defn}

  \begin{defn}\label{def_bicategory_of_spans}
    The bicategory of spans $W$ ...
  \end{defn}

  \begin{defn}\label{def_subcategories_of_epimorphisms_and_isomorphisms}
    subcategories of epimorphisms and isomorphisms of spans;
  \end{defn}

  \begin{defn}\label{def_lax_functor_V_to_eW}
    the lax functor from bicategory of spans to subcategory of epimorphisms
  \end{defn}

  \begin{defn}\label{def_functors_finite_sets_to_iW}
    functors from finite sets and finite sets op to $iW$: composition of inclusion into $V$ and $L$, actually lands in $iW$.
  \end{defn}

  \begin{defn}\label{def_functor_iW_to_Cat}
    functor from $iW$ to $Cat$ $\operatorname{Cat}$ ...
  \end{defn}

  \begin{defn}\label{def_pointed_stuff}
    pointed category, functor, smash, adjoining basepoint to category,
  \end{defn}

  \begin{defn}\label{def_added_basepoint_extending_functor}
    weak functor of bicategories, ...
  \end{defn}

  \begin{defn}\label{def_J_and_functors_from_finite_sets_to_pointed_categories}
    We define J and the two functors...
  \end{defn}

  \subsection{Left lax transformations, bimodules and homotopy colimits}

    \begin{defn}\label{def_left_lax_transformation}
      A left lax transformation ...
    \end{defn}

    \begin{rem}\label{rem_left_lax_transformation}
      remark about notation of 2-arrows in diagrams, axioms of a left lax 2-transf, ...
    \end{rem}

    \begin{rem}\label{rem_left_lax_transform_functors_J_to_Cat}
      Given two functors $E,F: J \to \Cat_*$ what does it mean to be a left lax transformation $E \Rightarrow F$?
    \end{rem}

    \begin{defn}\label{def_category_of_bimodules}
      category $\operatorname{Bimod}^J / E$ of bimodules (of $J$ over $E$?) consists of pairs $(F,G)$ ...
      % TODO find general definition for such bimodules?
    \end{defn}

    \begin{defn}\label{def_hocolim}
      The homotopy colimit ...
      % TODO find a good definition / model (i.e. source) to be used here
    \end{defn}
    % TODO do we need statements about the hocolim?

    \begin{rem}\label{rem_bimod_for_constant_functor_and_hocolim}
      Given a pointed small category $K$ with all small coproducts, we let $E = K$ be the constant functor $J \to \Cat_*$ ...
    \end{rem}

  \subsection{The left lax transformation $G^A$}

    \begin{defn}\label{def_left_lax_transformation_S}
      Let $\Sigma$ be the category of finite sets with bijections. We choose a strong symmetric monoidal functor $S: \Sigma \to \Top$
    \end{defn}

    NB We have to start working here!

    \begin{defn}\label{def_left_lax_transformation_A}
      Given a \hring $A$, we define a left lax transformation
    \end{defn}

    \begin{defn}\label{def_left_lax_transform_G^A}
      We combine the two above definitions to obtain a left lax transformation $G^A$ ...
    \end{defn}

  \subsection{Rectification}

    \begin{defn}\label{def_streets_first_construction}
      Street's first construction looks as follows; We take a lax functor $F: J \to \Cat_*$ ...
    \end{defn}

  \subsection{The Loday functor for finite sets}
    \begin{defn}\label{def_rectification_of_G^A}
      Applying Street's first construction to ...
    \end{defn}

    \begin{defn}\label{def_loday_functor_finite_sets}
      Given a finite set $S$ and a \hring $A$, we define the Loday functor of A at S to be ...
    \end{defn}

    \begin{lem}\label{lem_loday_functor_fixed_points}
      Can drop r for fixed points under finite group...
      \begin{proof}
        should be formal?
      \end{proof}
    \end{lem}

    \begin{lem}\label{lem_loday_functor_preserves_connectivity}
      The Loday functor (at a finite set) preserves connectivity of commutative ring spectra, sends stable equivalences to point-wise equivalences (check this statement!) and [comparison to smash product].
      \begin{proof}
        depends on connectivity of a map $S^1 \wedge A(S^n) \to A(S^{n+1})$ - corresponding map in orthogonal spectra context should be structure map. check definition of transformation $G^A$ or rather transformation $A$! Connection of connectivity to ``functors of simplicial sets''?
      \end{proof}
    \end{lem}

    \begin{cor}\label{cor_loday_at_S_naturally_equivalent_to_tensor_with_S}
      Let $A$ be a cofibrant? flat? \hring, then Loday of $A$ at $S$ is stably equivalent as a spectrum to $S \otimes A$, and the morphisms are natural in $S$.
    \end{cor}


    A bicategory $\mathcal{C}$ is in part made up of a class of 0-cells, and for any two zero-cells $A,B$ a category $\mathcal{C}(A,B)$, whose objects form the 1-cells from $A$ to $B$ and whose morphisms form the 2-cells between two given 1-cells. The bicategory of spans $W$ has 0-cells all finite cells. Given finite sets $X,Y$ the 1-cells are given as spans $ X \leftarrow A \rightarrow Y$ for some finite set $A$, and a 2-cell between two spans from $X$ to $Y$  is given as the vertical map in the following commutative diagram:
    \[
    \xymatrix@R-=.5em{
      %
      &
      %
      A \ar[dl] \ar[dd] \ar[dr]
      %
      \\
      X
      &
      %
      &
      Y
      \\
      %
      &
      A^\prime \ar[ul] \ar[ur]
      &
      %
    }
    \]
    Horizontal composition is given by a functorial and conrete choice of pullback applied to the 1-cells and taking the map induced by the 2-cells between pullbacks,

    \comm{[make clearer or scratch - this should explain horizontal composition of 2-cells!]}

    while vertical composition is composition of maps.\\
    The bicategory $\operatorname{Cat}$ of small categories has small categories as 0-cells, functors as 1-cells and natural transformations as 2-cells.
    \comm{add all the technical things you need from covering homology:}\\
    spans, functor $\cJ$, nat traf $G^A_S$ (gamma spaces, hom space (fibrant replacement), $(\Lambda_X A)^G$ functor of conn. comm $S$-algebras that preserves conn. and has values in very special gamma spaces (Cor. 5.1.5 in Covering homology), how diagonal is constructed (Street rectification necessary! H-set, and so on...); adapt to orthogonal spectra!?
    \begin{defn}\label{def_loday_functor}
      We define the Loday functor for a finite set $S$ and a commutative $\bS$-algebra $A$ as hocolim category functor ...

    \end{defn}

    \begin{defn}\label{def_loday_functor_in_symmetric_monoidal_category}
      Let $(\mathcal{C},\smash, \mathbb{1})$ be a symmetric monoidal category.\\
      \comm{TODO complete this}
    \end{defn}

    \begin{lem}\label{lem_loday_functor_is_simplicial}
      The Loday functor is a simplicial functor.\\
      \comm{[loday functor simplicial! add remark? add proof? add remark? source ($\Gamma$-spaces)?]}
    \end{lem}
