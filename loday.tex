% !TEX root = phd_thesis_krasontovitsch.tex

\section{Iterated THH - Loday Functor}
  We proceed to introduce topological Hochschild homology based on a space $X$ and the structure it carries for $X = T^n$ the $n$-dimensional torus, following \cite{brun2010covering} as well as \cite{carlsson2011higher}. For details on bicategories, confer \cite{benabou1967introduction}.\\
  \comm{[Should have two subsections here: spans + bicategories?]}

  \subsection{Spans of finite sets}

  \begin{defn}\label{def_cateogry_of_spans}
    The category of spans $V$ has as objects the class of finite sets, while the morphism set $V(Y,X)$ for two finite sets $X,Y$ is given by the set of equivalence classes of diagrams of the form $Y \ot A \to X$, called spans. Two such diagrams $Y \ot A \to X$, $Y \ot A^\prime \to X$ are said to be equivalent if there is a bijection $A \to A^\prime$ making the resulting triangles commutative. Composition of two maps represented by $[Y \ot A \to X]$ and $[X \ot B \to W]$ is given by taking the pullback of \comm{[complete this...]}
    % Since we take equivalence classes of morphisms, this is well-defined, and it is straight-forward to check that this forms a category.
  \end{defn}

  \begin{lem}\label{lem_coproduct_product_in_V}
    The product and coproduct in V are given by disjoint union.
    % \begin{proof}
      % Explain structure morphisms, comment on / explain universal properties
    % \end{proof}
  \end{lem}

  \begin{defn}\label{defn_group_acting_on_object_in_V_morphism_phi}
    Given a finite set $X$ with a left action of a group $G$, we define the action on $X$ through automorphisms in $V$ by mapping $g \mapsto [X \ot X \longrightarrow[g] X]$. % TODO fix arrows
  \end{defn}

  \begin{defn}\label{def_bicategory}
    A bicategory $\mathcal{C}$ consists of the following data:
  \end{defn}

  \begin{ex}\label{ex_bicategory_of_categories}
    The strict bicategory of small categories
  \end{ex}

  \begin{defn}\label{def_bicategory_functor}
    A functor $F: \mathcal{C} $% \mathcal{D}$ is a weak / strict / lax / ? functor...
  \end{defn}

  \begin{defn}\label{def_bicategory_of_spans}
    The bicategory of spans $W$ ...
  \end{defn}

  \begin{defn}\label{def_subcategories_of_epimorphisms_and_isomorphisms}
    subcategories of epimorphisms and isomorphisms of spans;
  \end{defn}

  \begin{defn}\label{def_lax_functor_V_to_eW}
    the lax functor from bicategory of spans to subcategory of epimorphisms
  \end{defn}

  \begin{defn}\label{def_functors_finite_sets_to_iW}
    functors from finite sets and finite sets op to $iW$: composition of inclusion into $V$ and $L$, actually lands in $iW$.
  \end{defn}

  \begin{defn}\label{def_functor_iW_to_Cat}
    functor from $iW$ to $Cat$ $\operatorname{Cat}$ ...
  \end{defn}

  \begin{defn}\label{def_pointed_stuff}
    pointed category, functor, smash, adjoining basepoint to category,
  \end{defn}

  \begin{defn}\label{def_added_basepoint_extending_functor}
    weak functor of bicategories, ...
  \end{defn}

  \begin{defn}\label{def_J_and_functors_from_finite_sets_to_pointed_categories}
    We define J and the two functors...
  \end{defn}

  \subsection{Left lax transformations, bimodules and homotopy colimits}

    \begin{defn}\label{def_left_lax_transformation}
      A left lax transformation ...
    \end{defn}

    \begin{rem}\label{rem_left_lax_transformation}
      remark about notation of 2-arrows in diagrams, axioms of a left lax 2-transformation, ...
    \end{rem}

    \begin{rem}\label{rem_left_lax_transform_functors_J_to_Cat}
      Given two functors $E,F: J \to \Cat_*$ what does it mean to be a left lax transformation $E \Rightarrow F$?
    \end{rem}

    \begin{defn}\label{def_category_of_bimodules}
      category $\operatorname{Bimod}^J / E$ of bimodules (of $J$ over $E$?) consists of pairs $(F,G)$ ...
      % TODO find general definition for such bimodules?
    \end{defn}

    \begin{defn}\label{def_hocolim}
      The homotopy colimit ...
      % TODO find a good definition / model (i.e. source) to be used here
    \end{defn}
    % TODO do we need statements about the hocolim?

    \begin{rem}\label{rem_bimod_for_constant_functor_and_hocolim}
      Given a pointed small category $K$ with all small coproducts, we let $E = K$ be the constant functor $J \to \Cat_*$ ...
    \end{rem}

  \subsection{The left lax transformation $G^A$}

    \begin{defn}\label{def_left_lax_transformation_S}
      Let $\Sigma$ be the category of finite sets with bijections. We choose a strong symmetric monoidal functor $S: \Sigma \to \Top$
    \end{defn}

    NB We have to start working here!

    \begin{defn}\label{def_left_lax_transformation_A}
      Given a \hring $A$, we define a left lax transformation
    \end{defn}

    \begin{defn}\label{def_left_lax_transform_G^A}
      We combine the two above definitions to obtain a left lax transformation $G^A$ ...
    \end{defn}

  \subsection{Rectification}

    \begin{defn}\label{def_streets_first_construction}
      Street's first construction looks as follows; We take a lax functor $F: J \to \Cat_*$ ...
    \end{defn}

  \subsection{The Loday functor for finite sets}
    \begin{defn}\label{def_rectification_of_G^A}
      Applying Street's first construction to ...
    \end{defn}

    \begin{defn}\label{def_loday_functor_finite_sets}
      Given a finite set $S$ and a \hring $A$, we define the Loday functor of A at S to be ...
    \end{defn}

    \begin{lem}\label{lem_loday_functor_fixed_points}
      Can drop r for fixed points under finite group...
      \begin{proof}
        should be formal?
      \end{proof}
    \end{lem}

    \begin{lem}\label{lem_loday_functor_preserves_connectivity}
      The Loday functor (at a finite set) preserves connectivity of commutative ring spectra, sends stable equivalences to point-wise equivalences (check this statement!) and [comparison to smash product].
      \begin{proof}
        depends on connectivity of a map $S^1 \wedge A(S^n) \to A(S^{n+1})$ - corresponding map in orthogonal spectra context should be structure map. check definition of transformation $G^A$ or rather transformation $A$! Connection of connectivity to ``functors of simplicial sets''?
      \end{proof}
    \end{lem}

    \begin{cor}\label{cor_loday_at_S_naturally_equivalent_to_tensor_with_S}
      Let $A$ be a cofibrant? flat? \hring, then Loday of $A$ at $S$ is stably equivalent as a spectrum to $S \otimes A$, and the morphisms are natural in $S$.
    \end{cor}


    A bicategory $\mathcal{C}$ is in part made up of a class of 0-cells, and for any two zero-cells $A,B$ a category $\mathcal{C}(A,B)$, whose objects form the 1-cells from $A$ to $B$ and whose morphisms form the 2-cells between two given 1-cells. The bicategory of spans $W$ has 0-cells all finite cells. Given finite sets $X,Y$ the 1-cells are given as spans $ X \leftarrow A \rightarrow Y$ for some finite set $A$, and a 2-cell between two spans from $X$ to $Y$  is given as the vertical map in the following commutative diagram:
    \[
    \xymatrix@R-=.5em{
      %
      &
      %
      A \ar[dl] \ar[dd] \ar[dr]
      %
      \\
      X
      &
      %
      &
      Y
      \\
      %
      &
      A^\prime \ar[ul] \ar[ur]
      &
      %
    }
    \]
    Horizontal composition is given by a functorial and conrete choice of pullback applied to the 1-cells and taking the map induced by the 2-cells between pullbacks, \comm{[make clearer or scratch - this should explain horizontal composition of 2-cells!]}, while vertical composition is composition of maps.\\
    The bicategory $\operatorname{Cat}$ of small categories has small categories as 0-cells, functors as 1-cells and natural transformations as 2-cells.
    \comm{add all the technical things you need from covering homology:}\\
    spans, functor $\cJ$, nat traf $G^A_S$ (gamma spaces, hom space (fibrant replacement), $(\Lambda_X A)^G$ functor of conn. comm $S$-algebras that preserves conn. and has values in very special gamma spaces (Cor. 5.1.5 in Covering homology), how diagonal is constructed (Street rectification necessary! H-set, and so on...); adapt to orthogonal spectra!?
    \begin{defn}\label{def_loday_functor}
      We define the Loday functor for a finite set $S$ and a commutative $\bS$-algebra $A$ as hocolim category functor ...

    \end{defn}

    \begin{defn}\label{def_loday_functor_in_symmetric_monoidal_category}
      Let $(\mathcal{C},\smash, \mathbb{1})$ be a symmetric monoidal category.\\
      \comm{TODO complete this}
    \end{defn}

    \begin{lem}\label{lem_loday_functor_is_simplicial}
      The Loday functor is a simplicial functor.\\

      \comm{[loday functor simplicial! add remark? add proof? add remark? source ($\Gamma$-spaces)?]}
    \end{lem}
