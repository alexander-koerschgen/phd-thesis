% !TEX root = phd_thesis_krasontovitsch.tex
\section{Higher de Rahm - Witt Complex}
%
%
\comm{[reorganize this whole beginning...!]}
\begin{thm}\cite[Thm. 3.22]{carlsson2011higher}
Fix an odd prime $p$. Let $A$ be a connective commutative ring spectrum. There is a pro multi-diffenrential graded ring \comm{say: higher/Burnside-Witt Complex?} $TR_q^\alpha(A) \defas TR_q^\alpha(A;p)$ consisting of the following data:
\begin{itemize}
 \item groups $TR^\alpha_q(A) \defas \pi_q ( T_{\sT_p^n}(A)^{L_\alpha} )$ for each $q \in \bZ$, $\alpha \in \cM_n$, where $L_\alpha \defas \alpha^{-1}(\bZ_p) / \bZ_p$,
 \item differentials $d_f: TR^\alpha_*(A) \to TR^\alpha_{*+k}(A)$ for each $f : S^k \to (\sT_p^n)_+$ in $\SHC$,
 \item \colorbox{yellow}{change range of maps for differentials! Use $C_k$!}
 \item restriction operators $R_\alpha \defas R^{(\beta)}_\alpha: TR^{\beta\alpha}_q(A) \to TR^\beta_q(A)$,
 \item Frobenius operators $F^\alpha \defas F^\alpha_{(\beta)}: TR^{\alpha\beta}_q(A) \to TR^\beta_q(A)$, and
 \item Verschiebung operators $V_\alpha \defas V_\alpha^{(\beta)}: TR^\beta_q(A) \to TR^{\alpha\beta}_q(A)$ for each $\alpha,\beta \in \cM_n$,
\end{itemize}
satisfying the following properties and relations:
\begin{itemize}
 \item the tripel $(TR^\alpha_q(A),d,R)$ forms a pro-multi differential graded ring;
 \item the differentials form an exterior algebra generated by $d_i$, $i \in \ind{n}$, where $d_i$ is induced by  $(e_i)_+ \circ \sigma: S^1 \to (\sT^1_p)_+ \to (\sT_p^n)_+$;
 \item the restriction maps are graded ring operators and commute with Frobenius, Verschiebung and differentials;
 \item the Frobenius maps are pro-graded ring operators;
 \item the Verschiebung maps are pro-graded [insert ring]-module operators\\%
    $V_\alpha: (F^\alpha)_* TR_*^\beta(A) \to TR_*^{\alpha\beta}(A)$;
 \item $F^\alpha = id$ and $V_\alpha = id$ for all $\alpha \in \Gl_n(\bZ_p)$;
 \item $R_{\alpha\beta} = R_\alpha R_\beta,$ $F^{\alpha\beta} = F^\beta F^\alpha,$ and $V_{\alpha\beta} = V_\alpha V_\beta$ for all $\alpha, \beta \in \cM_n$;
 \item $F^\alpha V_\beta = \abs{\gcd_{\alpha,\beta}} \cdot V_{[\lcm_{\alpha,\beta} / \alpha]}F^{[\lcm_{\alpha,\beta} / \beta]}$ %
 %$:TR^{[\lcm_{\alpha,\beta} / \beta] \gamma} \to TR^{[\lcm_{\alpha,\beta} / \alpha] \gamma}$
    for all $\alpha, \beta \in \cM_n$;
 \item $d_v F^\alpha = F^\alpha d_{\alpha v}$ and $V_\alpha d_v = d_{\alpha v} V_\alpha$ for all $v \in \bZ^n, \alpha \in \cM_n$;
 \item $F^\alpha d_v V_\beta = d_{\bez_{\alpha}\gcd_{\alpha,\beta}^\dagger v} %
    V_{[\lcm_{\alpha,\beta} / \alpha]} F^{[\lcm_{\alpha,\beta} / \beta]} + %
    V_{[\lcm_{\alpha,\beta} / \alpha]} F^{[\lcm_{\alpha,\beta} / \beta]}%
    d_{\bez_{\beta}\gcd_{\alpha,\beta}^\dagger v}.$
\end{itemize}
[INSERT: map $\lambda$ in zeroeth level from burnside witt ring, relation with frob, $\lambda$ and
differentials, meaning of gcd, lcm, dagger, absolute value aka volume, curly M sub n, remark on
when does the last composition make sense (find in GCD)]
\end{thm}
%
%
\begin{defn}\label{def_bwc}%burnside witt complex
\comm{add definition of burnside witt complex!, i.e. the above}\\
\comm{evaluated on homotopy groups}
\end{defn}
%
%
\begin{defn}\label{def_cmdga}
Let $n \geq 1$. A graded commutative $n$-multi-differential graded algebra $A^*$ is a graded commutative, graded algebra $(A^*)_{* \geq 0}$, together with additive maps $d_i: A^* \to A^{*+1}$ of degree $1$ for all $i \in \ind{n}$, satisfying the following properties:
\begin{itemize}
\item Each $d_i$ is a derivation, i.e. obeys the Leibniz rule (with Koszul sign): Let $x \in A^k, y \in A^l$, then for all $i \in \ind{n}$ we have%
$$d_i (xy) = d_i(x)y + (-1)^{k}xd_i(y);$$
\item The derivations anticommute with each other, i.e. for all $i,j \in \ind{n}$ we have%
$$d_i d_j = - d_j d_i.$$
\end{itemize}
These objects form a category, with morphisms given by maps of graded algebras commuting with the differentials, denoted $\cndga$.
%\\[2ex]
% \comm{QUESTIONS: }
% \comm{name/notation of $\bZ_p$-structure on differentials: seems a bit unnatural, any way to fix this!?}\\
% \comm{more elegant: forget basis; work with a) indexing morphism}\\
% \comm{b) torus homology c) exterior algebra over $\bZ_p$ ?}\\
% \comm{do we have more examples of this then our only one? bicomplex is an example for n=2}\\
\end{defn}
%
%
\begin{prop}\label{prop_higher_de_rahm_complex}
The functor%
$$(-)^0 : \cndga \to \calg$$%
$$A^* \longmapsto A^0,$$%
taking graded commutative n-multi-differential algebras to their zeroeth level, has a left adjoint%
$$\omg{-}{n}{*}: \calg \to \cndga,$$%
$$A \longmapsto \omg{A}{n}{*}$$
given on objects by the higher de-Rham complex, to be defined below.
\begin{proof}
We begin by constructing the higher de-Rham complex. Let $A$ be a commutative ring. We proceed as follows: In analogy to the classical de-Rham complex, we define ($n$ different!) derivations (and their compositions) as free $A$-modules, but without considering the Leibniz rule yet. We take the free graded commutative algebra over the graded module of all compositions of derivations, and finally divide out the Leibniz rule and it's higher analogue for iterated (i.e. compositions of) derivations. Finally we are able to define the differentials and prove the universal property.\\
% \comm{just some loud thking: differentials modelled after Kaehler diff}\\
% \comm{these are modules over the ground ring}\\
% \comm{top. diff are (in their argument) only linear over $\bZ$}\\
% \comm{proof of Fdw relation only for integral torus}\\
% \comm{hence indexing set for differentials is just $\Lambda^* \bZ^n$}\\
% \comm{[note: given input $A$ the differentials will be modules over $A$ but looking}\\
% \comm{[at all $A$, the structures are just over $\bZ$ or $\bZ_p$; rest governed by $F,V$,...]}\\
% \comm{[so yes $A$-modules, the construction in that sense is fine, you have $A$ in the bottom degree!]}\\
In the following, all tensor products are taken over the integers, i.e. $\otimes \coloneqq \otimes_{\bZ}$. Consider the inclusion of $A$-modules $A \otimes \bZ \to A \otimes A$, $a\otimes r \longmapsto a \otimes (r \cdot 1)$, where $A$ is acting on the left factor and $i \in A$ is the unit in $A$. Taking the cokernel yields a free A-module with elements denoted as $a\otimes b \eqqcolon a \del (b)$ such that the symbol $a \del (b)$ is bilinear in $(a,b)$ and $A$-linear in $a$, and further vanishes for $b \in \bZ$. Observe that this corrensponds to Kaehler differentials, without the Leibniz rule. Let
$$\tilde{D}^k \coloneqq (A \otimes A / A \otimes \bZ) ^ { I_{n,k} },$$
where $I_{n,k}$ is the set of injective maps $\ind{k} \to \ind{n}$. The collection $(D_k)_k$ is meant to model the space of $n$ different (alternating) differentials and their composites, or to be precise: composites of $k$ pairwise different differentials. We denote elements of $\tilde{D}^k$ living in the summand corresponding to $\alpha: \k \to \n$ as
\[a \del_\alpha (b) = a \del_{\alpha(k)} \ldots \del_{\alpha(1)} (b)\]%
for $a,b \in A$. Note that we choose to respect the order of composition, so we assume the indices ordered from right to left, hence they should be read from right to left.\\%
We define an $A$-linear action of $\Sigma_k = \{f: \ind{k} \to \ind{k} \with f \; \mathrm{bijective} \}$ on $\tilde{D}^k$ by setting%
$$\sigma \del_\alpha (a) = \sgn(\sigma) \cdot \del_{\alpha \sigma} (a)$$
and extending $A$-linearly, thus obtaining the $A$-module $D^k \coloneqq \tilde{D}^k/\Sigma_k$. This step should reflect the fact that differentials alternate with each other. Considering (the class of) some $\del_\alpha a \in D^k$ for $a \in A$, $\alpha \in I_{n,k}$ we will assume $\alpha$ to be strictly monotonously increasing unless noted otherwise.\\
We proceed by considering the graded module
\begin{equation*}
D \coloneqq \bigoplus_{k=1}^{n} D^k
\end{equation*}
where $D^k$ lives in grade $k$, and consider the free, graded commutative, graded algebra over $A$ on $D$, denoted $(\bigwedge D)^*$. It may be given as the free graded tensor algebra
\begin{equation*}
	T (D) = \sum_{i \geq 0} D^{\otimes i}
\end{equation*}
with multiplication given by concatenation, divided by the ideal generated by elements of the form
\begin{equation*}
	x \otimes y - (-1)^{\abs{x}\abs{y}} y \otimes x
\end{equation*}
for all $x,y \in D$. We will denote the class of an element $x \otimes y$ in $\bigwedge D$ as $x \wedge y$. We recall that the grading of an element $x = x_1 \wedge \ldots \wedge x_k \in \bigwedge D$ with $x_i \in D^{\abs{x_i}}$ for $i \in \ind{k}$ is given as the sum of the gradings of the factors:
\begin{equation*}
	\abs{x} = \sum_{i=1}^k \abs{x_i}.
\end{equation*}
We have a graded commutative ring, but in order to be able to define derivations $d_i: \bigwedge D ^* \to (\bigwedge D)^{*+1}$ for $i \in \ind{n}$ we must make sure that the relations implied by the Leibniz rule are valid. Hence we divide out the graded ideal $\mathcal{I}$ generated by the following elements: For all $x,y \in A$, $\alpha: \k \to \n$ strictly monotonously increasing we set $C \coloneqq \im \alpha \setminus B$ for $B \subseteq \im \alpha$ and take
\[ l(\alpha,x,y) \defas \del_\alpha(xy) - \sum_{B \subseteq \im \alpha} \sgn(\chi_{B,C}) %
  \del_B (x) \wedge \del_C (y) \in D^k \oplus ( \bigoplus_i D^i \wedge D^{k-i} ) \]
Note that we choose representatives with ordered indices, i.e. $\del_B= \del_{j_m} \cdots \del_{j_1}$ for $B = \{ j_1, \ldots, j_m \}$ with $j_1 < \ldots < j_m$, and analogously for $C$. Here the symbol $\chi_{B,C}$ refers to the shuffle permutation of $\im \alpha$ which orders $C$ before $B$; More precisely, let $\im \alpha = \{ i_1 \ldots i_k \}$ with $i_1 < \ldots < i_k$, $B = \{j_1 \ldots j_m \}$, $C = \{ l_1 \ldots l_{k-m} \}$ with $j_1 < \ldots < j_m$, $l_1 < \ldots < l_{k-m}$. Then $\chi_{B,C}$ permutes the tupel $(i_k \ldots i_1)$ to $(j_m \ldots j_1 \; l_{k-m} \ldots l_1)$.\\
We denote the quotient as%
\[ \omgn^* \coloneqq \bigmod {\left(\bigwedge D\right)} {\mathcal{I}} ^* \]%
and will subsequently drop the dimension and the ring from the notation unless there is the possibility of ambiguity.\\
Now, for $i \in \ind{n}$, which we fix, we may define additive $d_i: \bigwedge D^* \to \Omega^{*+1} $ and show that these descend to derivations ${d_i: \Omega^* \to \Omega^{*+1}}$. We proceed to define $d_i: \prod_{i=1}^k D \to \Omega^*$ using induction on $k$ and show first that this descends to $d_i: \bigwedge D^* \to \Omega^{*+1}$.\\
Since $d_i$ will only be additive, but not $A$-linear, for the empty product we set for $a \in A$:
\[
  d_i (a) = \del_i (a) \in D^1.
\]
For $a \del_\alpha (b) \in D^k$ we set
\[ d_i (a \del_\alpha (b)) \coloneqq  \del_i (a) \wedge \del_\alpha (b) + a \del_i \del_\alpha (b) \in D^1 \wedge D^k \oplus D^{k+1}, \]
where the second summand is defined to be zero if $i \in \im \alpha$. Both summands are $\bZ$-bilinear in $a$ and $b$ and vanish for $b \in \bZ$. Furthermore, the map is equivariant with respect to the $\Sigma_k$-action (acting on $\alpha$ on both sides). Hence this is well-defined and may be extended linearly. Note also that if $a \in \bZ$ (in particular if $a = 1$), the first summand vanishes.
\\
For a product with $k>1$ factors, let $x_i \in D^{\abs{x_i}}$, for $i \in \ind{k}$. Then $ x \defas ( x_1 , \ldots , x_k ) $ is sent to
    \[ d_i (x) \coloneqq%
    d_i (x_1) \wedge x_2 \wedge \ldots \wedge x_k + %
    (-1)^{\abs{x_1}} x_1 \wedge d_i (x_2, \ldots, x_k). \]%
Using the induction beginning and step we see that this is well-defined. Note also that the degree of each summand is indeed equal to the degree of $\abs{x}$ increased by one. Again, note that we only extend $\bZ$-linearly, so we shall define $d_i$ on formal products $a.x$ for $a \in A, \; x = (x_1 , \ldots , x_k) \in \prod_k D$. We set
\[
  d_i (a.x) = \del_i(a) \wedge %
  x_1 \wedge \ldots \wedge x_k + a \wedge d_i(x)
\]
and argue that this is well defined by showing that the definition is balanced with respect to $A$, i.e. descends to the tensor product (over $A$).\\
For the relations of the tensor product, the definition is multilinear (over $\bZ$) by definition and induction. To see that it respects the relation for graded commutativity, we need to take a closer look at the definition. Resolving the recursion, we arrive at the following formula for $x_1, \ldots, x_k \in D$:
\begin{equation*}
  d_i (x_1, \ldots , x_k) = %
  \sum_{i=1}^k %
  \left( \prod_{j=1}^{i-1} (-1)^{\abs{x_j}} \right) %
  x_1 \wedge \ldots \wedge d_i (x_i) %
  \wedge \ldots \wedge x_k,
\end{equation*}
and after staring at it for a moment one realizes that this is indeed alternating in the input, i.e. interchanging the position of say $x_l$ and $x_{l+1}$ leads to multiplication by the factor $(-1)^{\abs{x_l}\abs{x_{l+1}}}$.\\
The definition is balanced in $A$ by the following argument: For $a,b,c \in A$ we have
\begin{gather*}
  d_i ( a . (b \del_\alpha (c)) - %
    (ab) \del_\alpha (c) ) = \\ %
  \del_i(a) \wedge b \del_\alpha(c) + %
    a \wedge ( \del_i(b) \wedge \del_\alpha (c) + %
      b \del_i \del_\alpha (c) ) - %
    \del_i(ab) \wedge \del_\alpha (c) - %
    ab \del_i \del_\alpha (c) = \\%
  \del_i (a) \wedge b \wedge \del_\alpha (c) + %
    a \wedge \del_i (b) \wedge \del_\alpha (c) - %
    \del_i (ab) \wedge \del_\alpha (c) \in \cI,%
\end{gather*}
and
\begin{gather*}
  d_i(a.b - ab) = \del_i (a) \wedge b + a \wedge \del_i (b) - \del_i (ab) \in \cI
\end{gather*}
hence the above terms vanish in $\Omega^*$. As $d_i$ is alternating in its input, it suffices to show that it is $A$-linear, say, in the first variable. Due to the definition of $d_i$ (it satisfies the Leibniz rule) we see that $A$-linearity in the first variable follows from the above discussion: For $x_1 , \ldots , x_k \in D$, $a \in A$ we have
\begin{gather*}
  d_i (a. (x_1, \ldots, x_k) - (a x_1, x_2, \ldots, x_k)) = %
  d_i (a.x_1 - a x_1, x_2, \ldots , x_k) = \\ %
  d_i (a.x_1 - a x_1) \wedge x_2 \wedge \ldots \wedge x_k + (-1)^{\abs{x_1}} (a.x_1 - a x_1) \wedge d_i (x_2, \ldots , x_k) = 0 \in \Omega^* %
\end{gather*}
and hence $d_i$ indeed descends to a map $\bigwedge D ^* \to \Omega ^ {*+1}$.
\\
We show next that this also descends to a morphism $d_i: \Omega^* \to \Omega^{*+1}$. We have to check that for $x \in \mathcal{I}^*$ we have $d_i(x) \in \mathcal{I}^{*+1}$. Let $x,y \in A$, $\alpha \in I_{n,k}$ strictly monotonously increasing. We first show that $d_i ( l(x,y,\alpha) ) = \delta l(x,y,\tilde \alpha)$ for some $\delta \in \{1,-1\}$, where $\tilde \alpha: \ind{k+1} \to \n$ is the strictly monotonously increasing map with $\im \tilde \alpha = \im \alpha \cup \{i\}$. Note that we may assume $i \notin \im \alpha$, since otherwise we have $d_i ( l(x,y,\alpha) ) = 0 \in \cI^{*+1}$ by definition of $d_i$.\\
Recall that for $B \subseteq \im \alpha$ we write $C \defas \im \alpha \setminus B$. We compute:
    \[ d_i ( l(x,y,\alpha) ) =  d_i \left( \del_\alpha (xy) - \sum\limits_{B \subset%
    \im \alpha} \sgn(\chi_{B,C}) \del_B x \del_C y \right) = \]%
    %
    \[ \del_i \del_\alpha (xy) - \sum\limits_{B \subseteq \im \alpha} \sgn(\chi_{B,C})%
    d_i ( \del_B x \del_C y ) = \]%
    %
    \[ \del_i \del_\alpha (xy) - \sum\limits_{B \subseteq \im \alpha} \sgn(\chi_{B,C})%
    ( \del_i \del_B x \del_C y + (-1)^{\abs{B}} \del_B x \del_i \del_C y ). \]
Observe that all summands of $l(x,y,\tilde \alpha)$ appear, at least up to signs and permutations of differentials:
    \[ l(x,y,\tilde \alpha) = \del_{\tilde \alpha} (xy) - \sum_{B \subseteq \im \tilde \alpha} \sgn (\chi_{B,C})%
    \del_B x \del_C y. \]
Comparing the first two summands, we obtain
    \[ \del_{\tilde \alpha} (xy) = %
    (-1)^{\abs{\{ j \in \im \alpha \with j > i \}} } \del_i\del_\alpha (xy). \]
Hence it suffices to show that every other pair of summands also differs by this sign after permuting the indices so that they match. Let $N(\sigma)$ denote the number of inversions of a permutation $\sigma$. We have two cases to consider. Firstly, we have
    \[ \del_i \del_B x \del_C y = %
    (-1)^{ \{j \in B \with j > i\} } \del_{ B \cup \{i\} } x \del_C y,\]
so we want to show%
    \[ (-1)^{\abs{\{ j \in \im \alpha \with j > i \}} } \sgn(\chi_{B,C}) %
    (-1)^{\abs{\{ j \in B \with j > i \}} } = \sgn(\chi_{B \cup \{i\},C}),\]%
but
    \[ N(\chi_{B \cup \{i\},C}) = N(\chi_{B,C}) + %
    \abs{ \{ j \in C \with j > i\} }, \]%
and with%
    \[(-1)^{\abs{\{ j \in \im \alpha \with j > i \}} }  (-1)^{ \abs{ %
    \{ j \in B \with j > i \} } } = (-1)^{ \abs{ \{j \in C \with j > i\} } }\]%
(recall that $C = \im \alpha \setminus B$) we obtain equality. Secondly, we have%
    \[\del_B x \del_i \del_C y = (-1)^{ \abs{ \{j \in C \with j > i\} } }%
    \del_B x \del_{ C \cup \{i\} } y, \]
hence we are looking to show that%
    \[ (-1)^{\abs{\{ j \in \im \alpha \with j > i \}} } \sgn(\chi_{B,C})%
    (-1)^{\abs{B}} (-1)^{ \abs{ \{j \in C \with j > i\} } } = %
    \sgn( \chi_{ B, C \cup \{i\} } ), \]
which holds, as%
    \[ (-1)^{\abs{\{ j \in \im \alpha \with j > i \}} } %
    (-1)^{\abs{B}} (-1)^{ \abs{ \{j \in C \with j > i\} } } =%
    (-1)^{ \{ j \in B \with j < i \} } \]%
and%
    \[ N(\chi_{ B, C \cup \{i\} }) = N(\chi_{B,C}) + %
    \abs{ \{j \in B \with j < i\} }. \]
Hence we have shown that%
    \[ d_i ( l(x,y,\alpha) ) = (-1)^{ \{j \in \im \alpha \with j > i \} }%
    l(x,y,\tilde \alpha). \]
As $d_i$ is not multiplicative, but is additive, it now suffices to check that for all $r \in (\bigwedge D)^k$ and all generators $x \in \mathcal{I}^*$ we have $d_i(rx) \in \mathcal{I}^{k+*+1}$, but by definition of $d_i$ we have%
    \[ d_i(rx) = d_i(r) x + (-1)^k r d_i(x) \in \mathcal{I}, \]
since $\mathcal{I}$ is an ideal, and the claim follows: we have $d_i:\Omega^* \to \Omega^{*+1}$ for all $i \in \n$.\\
Finally we show that this construction is left-adjoint to evaluation at zero. From this, functoriality follows formally. Let $A \in \calg$, $B^* \in \cndga$, $\phi: A \to B^0$ a map of rings. We prove existence and uniqueness of a morphism of graded algebras $\phi^*: \Omega^*_{A,n} \to B^*$ with $\phi^0 = \phi$ and $\phi^{*+1} d_i = d_i \phi^*$ for all $i \in \n$. Following tradition we start with uniqueness: Assume we have a map as above. Then we have
    \[ \phi(a \del_{\alpha_1} (b_1) \wedge \ldots \wedge \del_{\alpha_m} (b_m)) = \]
    \[ \phi(a) \phi (\del_{\alpha_1} (b_1)) \ldots \phi(\del_{\alpha_m} (b_m)) = \]
    \[ \phi(a) d_{\alpha_1} (\phi(b_1)) \ldots d_{\alpha_m} (\phi(b_m)), \]
hence by using that $\phi$ is a ring homomorphism and that it commutes with the differentials, we see that such a morphism is uniquely determined by its values on $A$.\\ For the construction, consider%
    \[ \phi^k: \tilde D^k = (A \otimes A / A \otimes \bZ)^{I_{n,k}} %
    \to B^k, \; a \del_\alpha (b) \mapsto %
    \phi(a) d_\alpha (\phi(b)). \]
This is a well-defined morphism of $A$-modules, where $A$ acts on $B^k$ via the pullback $\phi^*:B^0-\mathrm{mod} \to A-\mathrm{mod}$, and constant on $\Sigma_k$-orbits, hence we get a map of $A$-modules%
    \[ \phi^*: D^* = \bigoplus_{i=1}^n D^i \to B^* \]
which in turn lets us define a ring homomorphism
    \[ \phi ^*: (\bigwedge D)^* \to B^*, %
    a \del_{\alpha_1} (b_1) \wedge \ldots \wedge \del_{\alpha_m} (b_m) \mapsto%
    \phi(a) d_{\alpha_1} (\phi(b_1)) \ldots d_{\alpha_m} (\phi(b_m)), \]
as the target is graded commutative and the source is the free graded commutative graded ring on $D^*$.
By Lemma \ref{lem_higher_leibniz_rel}, we have $\mathcal{I} \subseteq \ker \phi$, so we get an induced ring homomorphism%
\[\phi^*: \Omega^* = \Omega^*_{A,n} \to B^*.\]
Let us check that this commutes with the differentials. Let $i \in \n$. We proceed again by induction over the factors in $\Omega^*$: For $a,b \in A$, $\alpha \in I_{n,k}$ we have
    \[ d_i \phi^k ( a \del_\alpha (b) ) = d_i (\phi(a) d_\alpha(\phi(b)) ) = %
    d_i (\phi(a)) d_\alpha(\phi(b)) + \phi(a) d_i d_\alpha (\phi(b)), \]
and
    \[ \phi^{k+1} d_i (a \del_\alpha (b)) = %
    \phi^{k+1} (\del_i (a) \wedge \del_\alpha (b) + a \del_i \del_\alpha (b) ) = %
    d_i (\phi(a)) d_\alpha(\phi(b)) + \phi(a) d_i d_\alpha (\phi(b)). \]
Now let $a_j, b_j \in A$, $\alpha_j \in I_{n,k_j}$ for all $j \in \ind{m}$.%
Then%
    \[ d_i \phi^* ( a_1 \del_{\alpha_1} (b_1) \wedge \ldots \wedge %
    a_m \del_{\alpha_m} (b_m) ) = \]%
    %
    \[d_i ( \phi^{k_1}( a_1 \del_{\alpha_1} (b_1) ) \wedge \ldots \wedge %
    \phi^{k_m} ( a_m \del_{\alpha_m} (b_m) ) ) = \]%
    %
    \[d_i ( \phi^{k_1}( a_1 \del_{\alpha_1} (b_1) ) ) \wedge %
    \phi^{k_2}( a_2 \del_{\alpha_2} (b_2) ) \wedge \ldots \wedge %
    \phi^{k_m}( a_m \del_{\alpha_m} (b_m) ) + \]%
    \[ + (-1)^{k_1} \phi^{k_1}( a_1 \del_{\alpha_1} (b_1) ) \wedge %
    d_i ( \phi^{k_2}( a_2 \del_{\alpha_2} (b_2) ) \wedge \ldots \wedge %
    \phi^{k_m}( a_m \del_{\alpha_m} (b_m) ) ),
    \]
while (note that $*$ in the following is just a placeholder and may vary)
    \[
    \phi^{*} d_i ( a_1 \del_{\alpha_1} (b_1) \wedge \ldots \wedge %
    a_m \del_{\alpha_m} (b_m) ) = \]%
    %
    \[ \phi^{*} ( d_i ( a_1 \del_{\alpha_1} (b_1) ) \wedge a_2 \del_{\alpha_2} (b_2) \wedge \ldots \wedge a_m \del_{\alpha_m} (b_m) + \]%
    \[ + (-1)^{k_1} a_1 \del_{\alpha_1} (b_1) \wedge d_i (a_2 \del_{\alpha_2} (b_2) %
    \wedge \ldots \wedge a_m \del_{\alpha_m} (b_m)) ) = \]%
    %
    \[\phi^{k_1+1} (d_i ( a_1 \del_{\alpha_1} (b_1) ) ) \wedge %
    \phi^{k_2}( a_2 \del_{\alpha_2} (b_2) ) \wedge \ldots \wedge %
    \phi^{k_m}( a_m \del_{\alpha_m} (b_m) ) + \]%
    \[ + (-1)^{k_1} \phi^{k_1}( a_1 \del_{\alpha_1} (b_1) ) \wedge \phi^* d_i (a_2 \del_{\alpha_2} (b_2) \wedge \ldots \wedge a_m \del_{\alpha_m} (b_m)), \]
and again using the induction beginning as well as step we see that $\phi$ and $d_i$ commute, proving existence and hence the proposition.
\end{proof}
\end{prop}
%
%
\begin{lem}\label{lem_higher_leibniz_rel}
Let $A^* \in \cndga$, $\alpha:\k \to \n$ strictly monotonously increasing, and $x,y \in A^*$. For $B \subseteq \im \alpha$, set $C \defas \im \alpha \setminus B$. Then the following formula holds:%
    \[ d_\alpha (xy) = \sum_{B \subseteq \im \alpha} \sgn( \chi_{B,C} ) (-1)^{\abs{x}\abs{C}} d_B x d_C y. \]
\begin{proof}
We proceed by induction on $k$: Let $k \in \n$, then, as $d_k$ is a derivation, we have
\[ d_k(xy) = d_k(x) \, y + (-1)^{\abs{x} \cdot 1} x \, d_k(y). \]
Let $\alpha: \k \to \n$ as above, $\im \alpha = \{ \alpha_1 \ldots \alpha_k \}$, $\alpha_1 \less \ldots \less \alpha_k$, and $\alpha^\prime \defas \alpha|_{\k \setminus \{1\}}:\{2,\ldots,k\} \to \n$. We compute
\begin{gather*}
  d_\alpha (xy) = d_{\alpha^\prime}\,d_{\alpha_1}(xy) =%
    d_{\alpha^\prime}( d_{\alpha_1}(x) y + (-1)^{\abs{x}} x \, d_{\alpha_1}(y) ) =\\[.3em]
  \sum_{B^\prime \subseteq \im \alpha^\prime} \sgn(\chi_{B^\prime,C^\prime}) %
    (-1)^{ (\abs{x}+1) \abs{C^\prime} } d_{B^\prime} d_{\alpha_1}(x)d_{C^\prime}(y) \;\; + \\%
  (-1)^{\abs{x}} \sum_{B^\prime \subseteq \im \alpha^\prime}%
    \sgn(\chi_{B^\prime,C^\prime}) (-1)^{ \abs{x} \abs{C^\prime} } %
    d_{B^\prime}(x) \, d_{C^\prime}d_{\alpha_1}(y) = \\ %
  \sum_{B^\prime \subseteq \im \alpha^\prime} \sgn(\chi_{B^\prime,C^\prime}) %
    (-1)^{\abs{C^\prime}}(-1)^{ \abs{x} \abs{C^\prime} } %
    d_{B^\prime \cup \{\alpha_1\} }(x) d_{C^\prime}(y) \;\; + \\%
  \sum_{B^\prime \subseteq \im \alpha^\prime} \sgn(\chi_{B^\prime,C^\prime}) %
    (-1)^{\abs{x}(\abs{C^\prime}+1)} %
    d_{B^\prime}(x) d_{C^\prime \cup \{\alpha_1\}}(y) = \\%
  \sum_{B \subseteq \im \alpha} \sgn( \chi_{B,C} ) (-1)^{\abs{x}\abs{C}} d_B (x) d_C (y).
\end{gather*}
The first equalities are elementary manipulations and the induction hypothesis. For the last equality, we collected the two sums, noting that %
  $\sgn(\chi_{B^\prime,C^\prime})(-1)^{\abs{C^\prime}} = %
    \sgn(\chi_{B^\prime \cup \{\alpha_1\},C^\prime})$.
\end{proof}
\end{lem}
%
%
\begin{lem}\label{lem_freyds_adjoint_functor}\cite[V.6 Theorem 1]{mac1978categories}
Let $C$ be a small complete (i.e. all limits over small indexing categories exist) category. Then $C$ has an initial object if and only if it satisfies the solution set condition: There exists a set $I$ and an $I$-indexed family $(x_i)_{i \in I}$ in $C$ such that for every $c \in C$ there is an $i \in I$ and a morphism $x_i \to c$.
\end{lem}
%
%
\begin{thm}\label{thm_existence_initial_object}
There is an initial object in the category of Burnside-Witt complexes over $A$, called the de Rham-Burnside-Witt complex, and denoted $\cW_\cdot \Omega_{A,n}^*$.
\end{thm}
We follow the strategy of the proof of Theorem A in \cite{hesselholt2004rham}. Due to Freyd's adjoint functor theorem (cf. Lemma \ref{lem_freyds_adjoint_functor}), it suffices to show that we have a map%
\[ \tilde{\Omega}_{W_{\Cdot} A}^* \to E^*_{\Cdot} \]
out of a fixed object into any Burnside-Witt complex $E$ (over $A$), whose image is a Burnside-Witt subcomplex, for considering the isomorphism classes of these subcomplexes for varying $E$ we note that these form a set, as they are all quotients of $\tilde{\Omega}_{W_{\Cdot} A}$, giving the set and the morphisms (a subcomplex of every Burnside-Witt complex is represented) required by the solution set condition.\\
To this end, we extend the construction of the multi-differential de Rham complex $\Omega^*_{\Cdot} \defas \Omega^*_{W_{\Cdot} A}$ based on the Burnside Witt vectors to incorporate the Verschiebung operators. The structure of Burnside-Witt complexes will then guarantee that the image is a sub-complex. We begin with
%
%
\begin{defn}\label{def_admissible_words}
Consider the alphabet
\begin{equation*}
\cA \defas \{\cV_\alpha^{\beta\alpha},\; x_\gamma \with x \in \Omega^*_{\gamma}, \gamma \in \cM_n, * \geq 0, \alpha, \beta \in \cM_n,\},
\end{equation*}
where we abbreviate $x_\gamma \defas (x,\gamma)$ and $\cV_\alpha^{\beta\alpha} \defas (\cV,\alpha,\beta\alpha)$, and the monoid of all words (or strings) on this alphabet $\cS \defas \bigcup_{i \geq 0} \cA^i$ under concatenation, with unit the empty word $\varnothing \to \cA$. Here we attached the isogeny $\alpha$ to an $x \in \Omega^*_{\alpha}$ in order to distinguish elements corresponding to different isogenies $\alpha \neq \beta$ with $L_\alpha = L_\beta$ (and subsequently $W_\alpha A = W_{L_\alpha} A = W_{L_\beta} A = W_\beta A$).\\
We define a subset of admissible words together with a map
\begin{equation*}
  \cS_{\mathrm{adm}} \subseteq \cS,\;\; %
  \theta: \cS_{\mathrm{adm}} \to \cM_n,
\end{equation*}
where the former will pick out the sequences of symbols that we may interpret sensibly in our context, while the latter will indicate where they are to be interpreted. The definition is done according to the following rules:
\begin{itemize}
\item $x \in \Omega^*_{\alpha} \Rightarrow x_\alpha \in \cS_{\mathrm{adm}}$, $\theta(x_\alpha) \coloneqq \alpha$
\item $w,w^\prime \in \cS_{\mathrm{adm}}$, $\theta(w) = \theta(w^\prime) \Rightarrow ww^\prime \in \cS_{\mathrm{adm}}$, %
    $\theta(ww^\prime) \defas \theta(w)$
\item $w \in \cS_{\mathrm{adm}}, \beta \in \cM_n, \alpha \defas \theta(w) \Rightarrow \cV_\alpha^{\beta\alpha}w \in \cS_{\mathrm{adm}}$, %
      $\theta(\cV_\alpha^{\beta\alpha}w) \defas \beta\alpha$
\end{itemize}
Here, $ww^\prime$ is the concatenation of $w$ and $w^\prime$, and we define $\cS_\mathrm{adm}$ to be the set of all words that may be obtained by a finite number of applications of the above rules. We are capturing all of $\Omega^*_\cdot$, concatenation is a model for multiplication, and we are able to apply the Verschiebung operators, which really is our goal.

Proving that this is well-defined is a bit complicated. We formulate the following statements, which will let us show that $\theta$ and hence $\cS_{\mathrm{adm}}$ are well-defined, by induction over the length of words.  The complicated part is that we have to do an involved induction. We denote the length of a word $w \in \cS$ by $l(w)$, meaning the number of letters in $w$, or in symbols: $w \in \cA^i \Rightarrow l(w) = i$.
For all $n \geq 2$ and $m \geq 1$ we let
\begin{itemize}
\item[] \emph{A(n)}: For all $w \in \cS_{\mathrm{adm}}$ with $l(w) \leq n$, if we can write
	\begin{equation*}
	w = \cV^{\beta\alpha}_\alpha w^\prime,
	\end{equation*}
	then we may find 	$w^{\prime\prime}, w^{\prime\prime\prime} \in \cS_{\mathrm{adm}}, \theta(w^{\prime\prime}) = \alpha, \theta(w^{\prime\prime\prime}) = \beta\alpha$ (or $w^{\prime\prime\prime}$ is empty) such that
	\begin{equation*}
	w = \cV_\alpha^{\beta\alpha} w^{\prime\prime} w^{\prime\prime\prime}.
	\end{equation*}
\item[] \emph{B(n)}: For all $w = uv \in \cS_{\mathrm{adm}}$ with $l(w) \leq n$ we have that if $u$ is admissible and $\theta(u) = \alpha$, then also $\theta(w) = \alpha$.
\item[] \emph{WD(m)}: For all $w \in \cS_{\mathrm{adm}}$ with $l(w) \leq m$ we have that $\theta(w)$ is well-defined, i.e. all possible ways of assigning a value $\theta(w) \in \cM_n$ to $w$ coincide.
\end{itemize}
\end{defn}
%
%
\begin{lem}\label{lem_theta_well_defined}
The statements \emph{A(n)}, \emph{B(n)}, and \emph{WD(m)} are true for all $n \geq 2, m \geq 1$.
\begin{proof}
We first prove that $\mathrm{WD}(n)$ $\Rightarrow$ $\mathrm{A}(n)$ and that $(\mathrm{WD}(n), \mathrm{A}(n)) \Rightarrow \mathrm{B}(n)$ for all $n \geq 2$, proceed to prove $\mathrm{WD}(1)$ as well as $\mathrm{WD}(2)$, and finally prove that $\mathrm{WD}(n), \mathrm{A}(n)$, and $\mathrm{B}(n)$ together imply $\mathrm{WD}(n+1)$ for $n \geq 2$.

Let $n \geq 2$ and assume that $\theta$ is well-defined on all admissible words of length at most $n$. We prove $\mathrm{A}(n)$ by induction over $n = l(w)$ with $w = \cV^{\beta\alpha}_\alpha w^\prime \in \cS_{\mathrm{adm}}$.\\
Assuming the case $n = 2$ we have $w^\prime \in \cA^1 = \cA$. If $w^\prime = \cV^{\gamma\delta}_\delta$ for some $\gamma,\delta \in \cM_n$, $w = \cV^{\beta\alpha}_\alpha \cV^{\gamma\delta}_\delta$ would not be admissible, hence $w^\prime = x_\gamma$ for some $\gamma \in \cM_n$. But since $w$ is admissible, we must in particular have $\gamma = \alpha$, and we may set $w^{\prime\prime} \defas x_\gamma$ and let $w^{\prime\prime\prime}$ be empty.\\
Now let $n = l(w) > 2$. There are two cases to consider, corresponding to the second and third rule of defining the set of admissible words. Let $w = xy$ for some $x,y \in S_{\mathrm{adm}}$ with $\theta(x) = \theta(y)$. We have that $x = \cV^{\beta\alpha}_\alpha x^\prime$ for some $x^\prime \in \cS$, and since $y$ is not empty, $l(x) < l(w) = n$ and we may apply the induction hypothesis $\mathrm{A}(n-1)$ and obtain $x^{\prime\prime},x^{\prime\prime\prime} \in \cS_{\mathrm{adm}}$ with $\theta(x^{\prime\prime}) = \alpha$, $\theta(x^{\prime\prime\prime}) = \beta\alpha$ (or $x^{\prime\prime\prime}$ empty) and $x = \cV^{\beta\alpha}_\alpha x^{\prime\prime}x^{\prime\prime\prime}$. Since $\mathrm{WD}(n)$ holds, we may conclude that $\beta\alpha = \theta(x) = \theta(y)$ and may hence set $w^{\prime\prime} \defas x^{\prime\prime}$ and $w^{\prime\prime\prime} \defas x^{\prime\prime\prime} y$. Finally, let $w = \cV^{\gamma\delta}_\delta z$ with $z \in S_{\mathrm{adm}}, \theta(z) = \gamma\delta$. But then $\delta = \alpha, \gamma = \beta$ and we may set $w^{\prime\prime} \defas z$ and leave $w^{\prime\prime\prime}$ empty.

Assuming well-definedness for words of length up to $n$ as well as property $mathrm{A}(n)$, we proceed to prove the validity of $B(n)$.\\
Let $w = uv \in \cS_{\mathrm{adm}}$, $u \in \cS_{\mathrm{adm}}$ and $\theta(u) = \alpha$. We want to show that $\theta(w) = \alpha$ hols as well. Assume first that $l(w) = 2$, then $u = x_\gamma$ and, as above, we must have $w = y_\gamma$. Now assume $l(w) = n$ for some $n > 2$. Again, we differentiate by the two rules: Let $w = xy$ for $x,y \in \cS_\mathrm{adm}$ with $\theta(x) = \theta(y) \eqqcolon \gamma$. Since we assume $\mathrm{WD}(n)$ to be true, we may conclude $\theta(w) = \theta(x)$. Now there are two possibilities regarding the length of $x$: Either $l(u) \leq l(x)$, or $l(x) < l(u)$. In the first case, as $y$ is admissible, it is not empty, hence $l(x) < l(w) = n$ and we may apply $\mathrm{B}(n-1)$ to $x$ to conclude that $\alpha = \theta(u) = \theta(x) = \theta(w)$. For the second case, we may again apply $\mathrm{B}(n-1)$, but this time to $u$, and conclude that $\theta(w) = \theta(x) = \theta(u) = \alpha$. For the other rule, we assume $w = \cV_\gamma^{\delta\gamma} w^\prime$ for some $w^\prime \in \cS_{\mathrm{adm}}$ with $\theta(w^\prime) = \delta\gamma$. By well-definedness we have $\theta(w) = \delta$. We may write $u = \cV_\gamma^{\delta\gamma} u^\prime$ for some $u^\prime \in \cS$, and conclude by $\mathrm{A}(n)$ that there are $u^{\dprime},u^{\tprime}\cS_{\mathrm{adm}}$ with $\theta(u^{\dprime}) = \gamma, \; \theta(u^{\tprime}) = \delta\gamma$ (or $u^{\tprime}$ is empty). Arguing with $\mathrm{WD}(n)$ we see that $\alpha = \theta(u) = \delta\gamma = \theta(w)$, which concludes the proof of $\mathrm{B}(n)$.

Next, we set out to prove $\mathrm{WD}(1)$ and $\mathrm{WD}(2)$. This is fairly simple, as words of length one consist of letters $x_\gamma = (x,\gamma)$ and $(x,\gamma) \neq (x,\delta)$ for $\delta \neq \gamma$. For words $w = ab$ of length two with $a,b \in \cA$, assuming $b=\cV_\alpha^\beta\alpha$ for some $\alpha,\beta \in \cM_n$ leads to a contradiction to $w$ being admissible for any choice of $a \in \cA$, hence $b = x_\gamma$ for some $\gamma \in \cM_n$. If $a = \cV_\alpha^{\beta\alpha}$, we have, as before, that $\gamma = \alpha$, as $w$ is admissible. If $a = y_\delta$, for $w$ to be admissible we must have $\delta = \gamma$, which concludes $\mathrm{WD(2)}$.

For the third part we assume $\mathrm{WD}(n-1)$ for $n \geq 2$ and prove $\mathrm{WD}(n)$. By the above considerations, we may also assume $\mathrm{A}(n-1)$ as well as $\mathrm{B}(n-1)$. There are two cases to consider in which an ambiguity may arise: That an admissible word is obtained by an application of rule number two and three, or by two (possibly different) applications of rule number two. We shall show that both cases result in coinciding definitions of $\theta$ for that word.\\
Let $w \in \cS_\mathrm{adm}$ with $l(w)=n$. Assume first that $w = \cV_\alpha^{\beta\alpha} y = uv$ for $u,v,y \in \cS_\mathrm{adm}$ and $\theta(u) = \theta(v)$, $\theta(y) = \alpha$. Observe that $l(u) < l(w)$, as $v \in \cS_\mathrm{adm}$ is non-empty. Then $u = \cV_\alpha^{\beta\alpha} u^\prime$ for some $u^\prime \in \cS$ and we may apply $\mathrm{A}(n-1)$ to obtain $u = \cV_\alpha^{\beta\alpha} u^\dprime u^\tprime$ with $u^\dprime, u^\tprime \in \cS_\mathrm{adm}$, $\theta(u^\dprime) = \alpha$, $\theta(u^\tprime) = \beta\alpha$ (or $u^\tprime$ is empty), thus proving $\theta(u) = \beta\alpha$. Note that we also have $y = u^\dprime u^\tprime v \in \cS_\mathrm{adm}$, and by $\mathrm{B}(n-1)$ (and well-definedness of $\theta$) we may conclude that $\alpha = \theta(y) = \theta(u^\dprime) = \beta \alpha$. Hence this ambiguity may only arise in the case of using a letter of the form $\cV_\alpha^\alpha$, which does not lead to ambiguity with respect to $\theta$.\\
Assume finally that $w = uv = xy$ for $u,v,x,y \in \cS_\mathrm{adm}$ and $\theta(u) = \theta(v)$, $\theta(x) = \theta(y)$. We intend to prove that $\theta(u) = \theta(x)$. Again, we have two possibilities: Either $l(u) \leq l(x)$, or $l(x) < l(u)$. In the first case, according to $\mathrm{B}(n-1)$ we may conclude that $\theta(x) = \theta(u)$. Similarly, the second case implies $\theta(u) = \theta(x)$.
\end{proof}
\end{lem}
%
%
\begin{defn}
Now, for each $\alpha \in \cM_n$, we consider the submonoid $\cS_\alpha \defas \theta^{-1}(\alpha) \cup \{\varnothing\} \subseteq \cS_{\mathrm{adm}}$. Note that it is a graded submonoid, where a word $w \in \cS$ is graded using the grading of $\Omega^*_{\Cdot}$, i.e. $\abs{w} = \sum_{x_{\Cdot} \in w} \abs{x_{\Cdot}}$. We take the free abelian group with basis given by this submonoid, denoted
\begin{equation*}
  \hat\Omega^*_\alpha = \bigoplus\nolimits_{\cS_\alpha} \bZ.
\end{equation*}
and give it the structure of a graded ring, using concatenation of words $\cS_\alpha \times \cS_\alpha \to \cS_\alpha$ and extending bilinearly. This is also known as the (graded) monoid ring of $\cS_\alpha$ over $\bZ$, denoted as $\hat \Omega^*_\alpha = \bZ[\cS_\alpha]$.
\end{defn}
\begin{proof}[Proof of Theorem \ref{thm_existence_initial_object}]
Let $E^*_{\Cdot}$ be a Burnside-Witt complex over $A$, and let $\phi_{\Cdot}: W_\Cdot A \to E^0_{\Cdot}$ be the ring morphism from the Burnside-Witt vectors over $A$ to the degree $0$ part of the complex $E$. Recall that we are trying to find a morphism into every such $E$, out of a fixed object. From Prop.~\ref{prop_higher_de_rahm_complex} we know that we may extend $\phi_\Cdot$ to a morphism
\begin{equation*}
	\phi_\Cdot^*: \Omega^*_{\Cdot} = \Omega^*_{W_{\Cdot} A,n} \to E^*_{\Cdot}
\end{equation*}
of multidifferential algebras. We proceed to extend this to a map
\begin{equation*}
	\hat\phi_\Cdot^*: \hat\Omega^*_{\Cdot} \to E^*_{\Cdot}
\end{equation*}
and show that the collection of images of $\hat\phi_\Cdot^*$ for all $\Cdot \in \cM_n$ is a sub Burnside-Witt complex of $E$. We begin by defining the map on the monoid $\cS_\alpha$ (simultaneously for all $\alpha \in \cM_n$) by setting
\begin{itemize}
\item[] $x \in \Omega^*_{\alpha},\;\; \hat\phi_\alpha^*(x) \coloneqq \phi_\alpha^*(x)$,%
\item[] $w,w^\prime \in \cS_\alpha,\;\; \hat\phi_\alpha^*(ww^\prime) \coloneqq %
		\hat\phi_\alpha^*(w) \hat\phi_\alpha^*(w^\prime)$,%
\item[] $w \in \cS_\gamma, \beta \in \cM_n: \beta\gamma = \alpha,\;\;%
		\hat\phi_\alpha^*(\cV_\gamma^{\beta\gamma}w) \defas V_\gamma^{\beta\gamma} \hat\phi_\alpha^*(w),$
\end{itemize}
and sending the empty word to zero. We have to make sure that this is well-defined as a map of sets - by the second (and last) point, it is by definition a map of monoids. This may now be proven rather easily, using the statements and techniques of Lemma \ref{lem_theta_well_defined}: We proceed by induction over the length of a word $w \in \cS_\alpha$. For words of lenth $0$ and $1$, this follows immediately from the definition. Let $w \in \cS_\alpha$ with $l(w) = n \geq 2$. If $w = uv =xy$ for $u,v,x,y \in \cS_\mathrm{adm}$ and $\theta(u) = \theta(v)$, $\theta(x) = \theta(y)$, then $\theta(u) = \theta(x)$ (as we saw in the Lemma cited above) and $\hat\phi$ is well-defined on $w$, as it is well-defined on $u,v,x,$ and $y$ by induction. If $w = \cV_\alpha^{\beta\alpha} y$ for some non-empty $y \in \cS_\alpha$ and $\beta \neq \id$, no ambiguity is possible (as seen in the last part of the proof of the Lemma) and we have well-definedness by induction. If $\beta = \id$, then $\hat\phi_\alpha(\cV^\alpha_\alpha) = V^\alpha_\alpha \hat\phi_\alpha(y) = \hat\phi_\alpha(y)$ is again well-defined by induction.

We extend this map linearly to obtain a family of ring morphisms $\hat\phi_\alpha: \hat \Omega_\alpha \to E^*_\alpha$, ranging over $\alpha \in \cM_n$. We must prove that the sum of the images of this collection is a sub Burnside-Witt complex. We have to check that it has all the necessary structure: It is a subring, invariant under differentials and Verschiebung by construction. Note that it is also invariant under restriction operators, as they commute with all other operators. Again by construction it receives a map from the Burnside-Witt vectors, namely $\phi$. All that we must show is that it is invariant under Frobenius operators $F^\gamma=F_{\gamma\delta}^{\delta}$ for all $\gamma,\delta \in \cM_n$.

Frobenius are ring morphisms, hence it suffices to check words of the form $\cV^{\beta\alpha}_\alpha x_\alpha$ for some $\alpha,\beta \in \cM_n$ and $x\in \Omega^*_\alpha$, i.e. to show that $F^\gamma V_\beta x \in \im \hat \phi$, where $V_\beta = V_\alpha^{\beta\alpha}$. In the following arguments, a lot of indices will come up, and change. As the bookkeeping is rather tedious and completely pointless to this proof, we omit it and abuse notation (heavily) to keep the indices simple. We may use Frobenius reciprocity to obtain $F^\gamma V_\beta x = V_\beta F^\gamma x$ (with indices secretely changed) and proceed to analyze $F x$, leaving out the Verschiebung, as it suffices to show that $Fx$ is in the image. As $F$ is multiplicative, we may assume $x$, which is the image of an element in the multi-differential de~Rham complex, to consist of a single factor, i.e. $x = d_I (x^\prime)$ for some $x^\prime \in W_\Cdot A$ and $I \subset \ind{n} = \{1,\ldots,n\}$. Any element in the Burnside-Witt ring may be written as the sum of appropriate Verschiebung and Teichm\"uller morphism applied to its coordinates (cf. Rem. \ref{rem_teichmueller_sum}), namely
\begin{equation*}
	x^\prime = \sum_\alpha V_\alpha \Delta_\alpha (x^\prime_\alpha)
\end{equation*}
for appropriately chosen $\alpha \in \cM_n$ (again, what these exactly are matters not to this proof), $x^\prime_\alpha \in A$. Both the differential and Frobenius are additive, hence we may reduce to considering
\begin{equation*}
F^\gamma d_I V_\alpha \Delta_\alpha (x)
\end{equation*}
for $x \in A$.

For the next reduction, we must analyze terms of the form $F^\alpha d_I V_\beta$ for $\alpha,\beta \in \cM_n$ and $I \subset \ind{n}$. The appropriate relation for this situation is terribly involved to spell out (cf. Lem. \ref{lem_rel_FdV_higher_differentials}), but ignoring indices, it becomes rather pleasant:
\begin{equation*}
F d_I V = \sum_{A \subset I} d_A VF d_B
\end{equation*}
where $B \coloneqq I \setminus A$. Again we may ignore the outer differential and Verschiebung and finally arrive at $F d_I \Delta (a)$ for some $a \in A$. But, according to Cor. \ref{cor_fdw_relation_arbitrary_dimensions}, we may rewrite this in a form where no Frobenius operator is present, which proves that the image of $\hat\phi$ is closed under Frobenius operators, and hence concludes the proof of the theorem.
\end{proof}
%
%
\begin{lem}
The map
\begin{equation*}
	\hat \phi_\Cdot: \hat \Omega^*_\Cdot \to \cW_\Cdot \Omega^*_{A,n}
\end{equation*}
is surjective.
\begin{proof}
The argument is formal and standard category theory: As the image of the above map is a Burnside-Witt complex, it receives a map from the initial object. The composition of this map with the inclusion is a self-map of the initial object, so it must be the identity, hence the inclusion is surjective.
\end{proof}
\end{lem}
\comm{Morten, you may stop reading here, if you wish: what follows are mere notes.}\\
\comm{[put the rest in different statement! relations that hold in initial object]}
For each $\alpha,\beta \in \cM_n$ we define a morphism \comm{[necessary?]} of graded abelian groups
\begin{equation*}
	V_\alpha^{\beta\alpha}: \hat \Omega_\alpha^* \to \hat \Omega_{\beta\alpha}^*
\end{equation*}
by sending a word
$w \in \cS_\alpha$ to $V_\alpha^{\beta\alpha}(w) \defas \cV_\alpha^{\beta\alpha}w \in \cS_{\beta\alpha}$ and extending this linearly. Finally, \\
We now have a formal ring structure, Verschiebungs operators and $\Omega^*_\alpha$. We should identify the formal ring structure with the ring structure already present on $\Omega^*_{\alpha}$. To this end we take the graded ideal $\cI_\alpha^* \subseteq \hat \Omega_\alpha^*$ generated by the following elements:\\
\comm{[divide into what relations are needed for proof of theorem, and rest]}\\
\comm{[rest being which other relations we can divide out]}
\begin{itemize}
%\item Given $w=x, w^\prime = x^\prime \in \cS_\alpha$ for $x, x^\prime \in \Omega^*_{W_\alpha A}$ we take $w + w^\prime - v$, where $w \defas y \in \cS_\alpha$ with $y \defas x + x^\prime \in \Omega^*_{W_\alpha A}$;
\item $(w + w^\prime - v) \in \hat\Omega_\alpha^*$ for all $w = x_\alpha$, $w^\prime =%
  x^\prime_\alpha$, $v = (x+x^\prime)_\alpha \in \cS_\alpha$ with $x, \; x^\prime \in%
  \Omega^*_{\alpha},$
\item $(ww^\prime - v) \in \hat\Omega_\alpha^{k+l}$ for all $w = x_\alpha$, $w^\prime =%
  x^\prime_\alpha$, $v = (x \cdot x^\prime)_\alpha \in \cS_\alpha$, with $x \in %
  \Omega^k_{\alpha}$, $x^\prime \in \Omega^l_{\alpha},$
\end{itemize}
and consider the quotient $\tilde \Omega^*_{\alpha} \defas (\hat \Omega_\alpha / \cI) ^*$. This quotient admits a map
\begin{equation}
  \Omega^*_{\Cdot}
\end{equation}
For (extended) functoriality of the Verschiebung operators, we include %
\begin{itemize}
\item $(\cV_{\gamma\beta}^{\delta\gamma\beta} \cV_{\beta}^{\gamma\beta} w - \cV_{\beta}^{\delta\gamma\beta} w ) \in \hat \Omega^*_\alpha$ for all
$w \in S_\beta, \beta,\gamma, \delta \in \cM_n, \delta\gamma\beta\ = \alpha$
\end{itemize}
on the one hand, while taking care of $V_\beta^{\gamma\beta} = \id$ for invertible $\gamma \in \cM_n$ on the other hand is a bit more involved:
Let $\beta, \gamma \in \cM_n$ with $\gamma\beta = \alpha$ and $\gamma$ invertible. We introduce an equivalence relation on $\cS_{\alpha}$ by defining $\hat w \in \cS_{\alpha}$ for each $w \in \cS_\beta$ and setting $\cV_\beta^{\gamma\beta} w \sim \hat w$, using induction on the length $l(w) \geq 1$ of $w$:
Let $w = x_\beta$ for some $x \in \Omega^*_{W_\beta A}$, then $\hat w \defas x_{\gamma\beta} = x_\alpha \in \cS_\alpha$. Note that since $\gamma$ is invertible, we have $L_\beta = L_{\gamma\beta}$, $W_\beta A = W_{L_\beta} A = W_{L_{\gamma\beta}} A  = W_{\gamma\beta} A$ and finally $\Omega^*_{W_\beta A} = \Omega^*_{W_{\gamma\beta} A}$, making this well-defined.\\
Now let $w \in \cS_\beta$ be of length at least two. We differentiate two cases: Let %
$w = \cV_\delta^{\epsilon\delta} v$ for $v \in \cS_\delta$, $\delta, \epsilon \in \cM_n$ with $\epsilon\delta = \beta$. Then we set $ \cV_\beta^{\gamma\beta} w = %
  \cV_\beta^{\gamma\beta} \cV_\delta^{\epsilon\delta} v \sim %
  \cV_\delta^{\gamma\epsilon\delta} v$. %
If $\epsilon$ and hence $\gamma\epsilon$ is not invertible, we set $\hat w \defas \cV_\delta^{\gamma\epsilon\delta} v \in \cS_{\alpha}$. If $\epsilon$ is invertible, then so is $\gamma\epsilon$, and as $v$ is shorter than $w$, by induction there is $\hat v \in \cS_{\gamma\epsilon\delta} = \cS_{\alpha}$ with $\cV_\delta^{\gamma\epsilon\delta} v \sim \hat v$, and we set $\hat w \defas \hat v$.\\
For the other case, assume that $w = uv$ for some $u,v \in \cS_\beta$. As $l(u) < l(w)$ and $l(v) < l(w)$, by induction there are $\hat u, \hat v \in \cS_{\gamma\beta} = \cS_\alpha$ as above, and we set $\hat w \defas \hat u \hat v$. Thus we include in the ideal the elements
\begin{itemize}
\item $(\cV_\beta^{\gamma\beta} w - \hat w ) \in \hat \Omega^*_{\alpha}$ for all %
  $\beta, \gamma \in \cM_n$ with $\gamma\beta = \alpha$ and $\gamma$ invertible, $w \in \cS_\beta$, $\hat w \in \cS_{\gamma\beta} = \cS_\alpha$ as defined above.
\end{itemize}
We go further and interpret the symbols $\cV$ in degree $0$, adding
\begin{itemize}
\item $(\cV_\beta^{\gamma\beta} x - y) \in \hat \Omega^0_{\alpha}$ for all %
  $\beta, \gamma \in \cM_n$ with $\gamma\beta= \alpha$, $x \in \Omega^0_{W_\beta A} = W_\beta A$, %
  where $y = V_\beta^{\gamma\beta} (x) \in W_{\alpha} A$.
  %BAZINGA interpret in higher degrees? use relation Vd = dV?
\end{itemize}
Finally, we make the multiplication graded commutative:
\begin{itemize}
\item $(w w^\prime - (-1)^{kl} w^\prime w) \in \hat\Omega^{k+l}_\alpha$ for all $w \in \hat\Omega^k_\alpha, w^\prime \in \tilde\Omega^l_\alpha$
\end{itemize}
We let all these elements generate the graded ideal $\cI_\alpha^*$. We want the relations to be closed under concatenation as well as under applying the Verschiebung operators. To this end, we need to iterate concatenation and Verschiebung as follows: Define an ascending sequence of graded ideals $\hat\cI_\alpha^{*\,(n)} \subseteq \hat\cI_\alpha^{*\,(n+1)}$ for all $n \geq 0$ by setting $\hat\cI_\alpha^{*\,(0)} \defas \cI_\alpha^*$ and letting $\hat\cI_\alpha^{*\,(n+1)}$ be the graded ideal generated by%
    \[\bigcup_{\beta,\gamma} V_\beta^{\gamma\beta}(\hat\cI_\beta^{*\,(n)}),\]%
where the union is taken over all $\beta,\gamma \in \cM_n$ with $\gamma\beta=\alpha$. We collect these into $\hat\cI_\alpha^{*} \defas \bigcup_{n \geq 0} \hat\cI_\alpha^{*\,(n)}$, to at last obtain, for each $\alpha \in \cM_n$. the graded commutative graded ring%
    \[\tilde\Omega_\alpha^* \defas \hat\Omega_\alpha^* / \hat\cI_\alpha^*.\]
